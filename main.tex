\documentclass[ucs,8pt]{beamer}

% Copyright 2004 by Till Tantau <tantau@users.sourceforge.net>.
%
% In principle, this file can be redistributed and/or modified under
% the terms of the GNU Public License, version 2.
%
% However, this file is supposed to be a template to be modified
% for your own needs. For this reason, if you use this file as a
% template and not specifically distribute it as part of a another
% package/program, I grant the extra permission to freely copy and
% modify this file as you see fit and even to delete this copyright
% notice.
%
% Modified by Tobias G. Pfeiffer <tobias.pfeiffer@math.fu-berlin.de>
% to show usage of some features specific to the FU Berlin template.

% remove this line and the "ucs" option to the documentclass when your editor is not utf8-capable
\usepackage[utf8]{inputenc}    % to make utf-8 input possible
\usepackage[brazil]{babel}
%\usepackage[english]{babel}     % hyphenation etc., alternatively use 'german' as parameter

% figure numbers, captions and subcaptions
\usepackage[font=footnotesize,format=plain,labelfont=bf,up,textfont=it,up,compatibility=false]{caption}
\usepackage[font=scriptsize,compatibility=false]{subcaption}
\usepackage[caption=false]{subfig}


\setbeamertemplate{caption}[numbered]

% links
%\usepackage[pdftex,plainpages=false,pdfpagelabels,pagebackref,colorlinks=true,citecolor=DarkGreen,linkcolor=NavyBlue,urlcolor=DarkRed,filecolor=Green,bookmarksopen=true]{hyperref}
\usepackage{hyperref}
%\pdfcatalog{ /PageMode /FullScreen }

% bibliography
\usepackage[fixlanguage]{babelbib}
%\usepackage{biblatex}

% % links coloridos
% \usepackage[all]{hypcap}                    % soluciona o problema com o
% hyperref e capitulos
\usepackage[round,sort,nonamebreak]{natbib} % citação bibliográfica textual(plainnat-ime.bst)
\bibpunct{(}{)}{;}{a}{\hspace{-0.7ex},}{,} % estilo de citação. Veja alguns exemplos em http://merkel.zoneo.net/Latex/natbib.php


\usepackage[sanitize=none,acronym,toc]{glossaries}
%\usepackage[acronym,toc]{glossaries}
% Define a new glossary type
\newglossary[slg,toc]{symbols}{sym}{sbl}{Lista de Símbolos}
\newglossary[elg]{equations}{eqn}{eql}{Equations}
\makeglossaries


% ---------------------------------------------------------------------------- %
% Dicionario de Termos:
\input{d01-terms}       
\input{d02-symbols}       
\input{d03-acronyms}       
\input{d04-equations}       
% ---------------------------------------------------------------------------- %


\graphicspath{{./images/}}             % caminho das figuras (recomendável)
% Template for talks using the Corporate Design of the Freie Universitaet
%   Berlin, created following the guidelines on www.fu-berlin.de/cd by
%   Tobias G. Pfeiffer, <tobias.pfeiffer@math.fu-berlin.de>
% This file can be redistributed and/or modified in any way you like.
%   If you feel you have done significant improvements to this template,
%   please consider providing your modified version to
%   https://www.mi.fu-berlin.de/w/Mi/BeamerTemplateCorporateDesign

\usepackage{amsmath,dsfont,listings}

%%% FU logo
% small version for upper right corner of normal pages
\pgfdeclareimage[height=0.9cm]{university-logo}{images/logo_emap}
\logo{\pgfuseimage{university-logo}}
% large version for upper right corner of title page
\pgfdeclareimage[height=1.0cm]{big-university-logo}{images/logo_emap}
\newcommand{\titleimage}[1]{\pgfdeclareimage[height=6cm]{title-image}{#1}}
\titlegraphic{\pgfuseimage{title-image}}
%%% end FU logo

% NOTE: 1cm = 0.393 in = 28.346 pt;    1 pt = 1/72 in = 0.0352 cm
\setbeamersize{text margin right=3.5mm, text margin left=7.5mm}  % text margin

% colors to be used
\definecolor{text-grey}{rgb}{0.45, 0.45, 0.45} 	% grey text on white background
\definecolor{bg-grey}{rgb}{0.66, 0.65, 0.60} 	% grey background (for white text)
\definecolor{usp-blue}{RGB}{16, 148, 171} 			% blue text
\definecolor{usp-blue-light}{RGB}{100, 196, 208} 	% blue text
\definecolor{usp-orange}{RGB}{252, 180, 33} 		% Orange text
\definecolor{usp-red}{RGB}{204, 0, 0} 			% red text (used by \alert)

% switch off the sidebars
% TODO: loading \useoutertheme{sidebar} (which is maybe wanted) also inserts
%   a sidebar on title page (unwanted), also indents the page title (unwanted?),
%   and duplicates the navigation symbols (unwanted)
\setbeamersize{sidebar width left=0cm, sidebar width right=0mm}
\setbeamertemplate{sidebar right}{}
\setbeamertemplate{sidebar left}{}
%    XOR
% \useoutertheme{sidebar}

% frame title
% is truncated before logo and splits on two lines
% if neccessary (or manually using \\)
\setbeamertemplate{frametitle}{%
    \vskip-30pt \color{text-grey}\large%
    \begin{minipage}[b][23pt]{80.5mm}%
    \flushleft\insertframetitle%
    \end{minipage}%
}

%%% title page
% TODO: get rid of the navigation symbols on the title page.
%   actually, \frame[plain] *should* remove them...
\setbeamertemplate{title page}{
% upper right: FU logo
\vskip2pt\hfill\pgfuseimage{big-university-logo} \\
\vskip-20pt\hskip3pt
% title image of the presentation
\begin{minipage}{5.0cm}
\hspace{-.1cm}\inserttitlegraphic
\end{minipage}

% set the title and the author
\vskip-12pt
\parbox[top][1.35cm][c]{\textwidth}{\raggedleft{\color{text-grey}\inserttitle \\ \small \insertsubtitle}}
\vskip3pt
\parbox[top][1.35cm][c]{\textwidth}{\raggedleft{\small \insertauthor \\ \insertinstitute \\[3mm]
\scriptsize \insertdate}} }
%%% end title page

%%% colors
\usecolortheme{lily}
\setbeamercolor*{normal text}{fg=black,bg=white}
\setbeamercolor*{alerted text}{fg=usp-red}
\setbeamercolor*{example text}{fg=usp-orange}
\setbeamercolor*{structure}{fg=usp-blue}

\setbeamercolor*{block title}{fg=white,bg=black!50}
\setbeamercolor*{block title alerted}{fg=white,bg=black!50}
\setbeamercolor*{block title example}{fg=white,bg=black!50}

\setbeamercolor*{block body}{bg=black!10}
\setbeamercolor*{block body alerted}{bg=black!10}
\setbeamercolor*{block body example}{bg=black!10}

\setbeamercolor{bibliography entry author}{fg=usp-blue}
% TODO: this doesn't work at all:
\setbeamercolor{bibliography entry journal}{fg=text-grey}

\setbeamercolor{item}{fg=usp-blue}
\setbeamercolor{navigation symbols}{fg=text-grey,bg=bg-grey}
%%% end colors

%%% headline
\setbeamertemplate{headline}{
\vskip4pt\hfill\insertlogo\hspace{3.5mm} % logo on the right

\vskip6pt\color{usp-blue}\rule{\textwidth}{0.4pt} % horizontal line
}
%%% end headline

%%% footline
\newcommand{\footlinetext}{\insertshortinstitute, \insertshorttitle, \insertshortdate}
\setbeamertemplate{footline}{
\vskip5pt\color{usp-blue}\rule{\textwidth}{0.4pt}\\ % horizontal line
\vskip2pt
\makebox[123mm]{\hspace{7.5mm}
\color{usp-blue}\footlinetext
\hfill \raisebox{-1pt}{\usebeamertemplate***{navigation symbols}}
\hfill \insertframenumber}
\vskip4pt
}
%%% end footline

%%% settings for listings package
\lstset{extendedchars=true, showstringspaces=false, basicstyle=\footnotesize\sffamily, tabsize=2, breaklines=true, breakindent=10pt, frame=l, columns=fullflexible}
\lstset{language=Python} % this sets the syntax highlighting
\lstset{mathescape=true} % this switches on $...$ substitution in code
% enables UTF-8 in source code:
\lstset{literate={ä}{{\"a}}1 {ö}{{\"o}}1 {ü}{{\"u}}1 {Ä}{{\"A}}1 {Ö}{{\"O}}1 {Ü}{{\"U}}1 }
%%% end listings  % THIS is the line that includes the template!

\usepackage{arev,t1enc} % looks nicer than the standard sans-serif font
% if you experience problems, comment out the line above and change
% the documentclass option "9pt" to "10pt"


% ---------------------------------------------------------------------------- %
\DeclareMathOperator*{\argmin}{arg\,min}
\DeclareMathOperator*{\argmax}{arg\,max}
\DeclareMathOperator*{\erf}{erf}
% ---------------------------------------------------------------------------- %


% image to be shown on the title page (without file extension, should be pdf or png)

\titleimage{images/capa_pga}

\title[PSHAB Salvador] % (optional, use only with long paper titles)
{PSHA-B 2014.02 \\~}

\subtitle{II Encontro do Grupo de Pesquisa \\ 
em Ameaça e Engenharia Sísmica no Brasil, \\
Salvador, BA}

\author[Pirchiner, Marlon] % (optional, use only with lots of authors)
{M.~Pirchiner }
% - Give the names in the same order as the appear in the paper.

\institute[IAG-USP / EMAp-FGV] % (optional, but mostly needed)
{\scriptsize{Centro de Sismologia-USP / EMAp-FGV}}
% - Keep it simple, no one is interested in your street address.

\date[ Setembro, 2014] % (optional, should be abbreviation of conference name)
{setembro de 2014}
%{Defesa de Mestrado, julho de 2014}
% - Either use conference name or its abbreviation.
% - Not really informative to the audience, more for people (including
%   yourself) who are reading the slides online

\subject{Seismology, Earthquake, Seismic Hazard, Smoothing}
% This is only inserted into the PDF information catalog. Can be left
% out.

% you can redefine the text shown in the footline. use a combination of
% \insertshortauthor, \insertshortinstitute, \insertshorttitle, \insertshortdate, ...
\renewcommand{\footlinetext}{\insertshortinstitute,\insertshorttitle,\insertshortdate}

% Delete this, if you do not want the table of contents to pop up at
% the beginning of each subsection:
\AtBeginSubsection{
  \begin{frame}<beamer>{Agenda}
  	\scriptsize{
 	   \tableofcontents[currentsection, currentsubsection]
  	}
  \end{frame}
  }


\begin{document}


\begin{frame}[plain]
  \titlepage
\end{frame}

\begin{frame}{Agenda}
 	\scriptsize{
   		\tableofcontents
 	}
 % You might wish to add the option [pausesections]
\end{frame}


% ====================================================
\section{9:00 - Boas-vindas}
% ====================================================

	\subsection{PSHAB: Grupo de Pesquisa em Ameaça e Engenharia Sísmica no Brasil}
	\begin{frame}{PSHAB: Grupo de Pesquisa em Ameaça e Engenharia Sísmica no Brasil}
		- PSHAB: Grupo de Pesquisa em Ameaça e Engenharia Sísmica no Brasil
		
		- mas nem o nome é concenso: PSHAB, PSHB, PSHAB ?!
	\end{frame}


	\subsection{Informes do último encontro (02/2014, RJ)}
	\begin{frame}{Informes do último encontro (02/2014, RJ)}
		- Informes do último encontro (02/2014, RJ)
	\end{frame}

	
	\subsection{Meta: artigo}
	\begin{frame}{Meta: artigo}
		- \textbf{Meta}: artigo discutindo a construção do modelo de ameaça sísmica
		para Brasil.
	\end{frame}


% ====================================================
\section{9:10 - Motivação}
% ====================================================

	\subsection{ABNT}
	\begin{frame}{Comissão ABNT para Estruturas Resistentes aos Sismos}
		- Comissão ABNT para Estruturas Resistentes aos Sismos
	\end{frame}
	
	\subsection{Modelos sobre o Brasil}
	\begin{frame}{Modelos sobre o Brasil}
		- Giardini(GSHAP).	
		
		- Petersen2010(preliminar)
	\end{frame}




% ====================================================
\section{9:30 - Dados públicos disponiveis}
% ====================================================

	\subsection{Catálogos: BSB2013.08 e BSB2014.06}
	\begin{frame}{Catálogos com ao menos uma magnitude Mw ?!}
		- BSB2013.08.	
		
		- BSB2014.06
		
		- com ao menos uma magnitude Mw ?!
	\end{frame}


	\subsection{Falhas neotectônicas}
	\begin{frame}{Falhas neotectônicas 'ameaçadoras'}
		- Existem dados públicos?!
		
		- Saad ?!
		
		- Hilário e/ou Ricomini ?!
	\end{frame}

	\subsection{Deformações crustais}
	\begin{frame}{Deformações crustais}
		
		- Existem dados públicos !?	
		
		- SIRGAS !?
		
		- Marota et al?!
	\end{frame}


% ====================================================
\section{9:50 - Aspectos metodologicos (OpenQuake)}
% ====================================================

	\subsection{i. fontes sismogênicas} 

		\subsubsection{diferentes abordagens para caracterização de fontes sismogênicas}
		\begin{frame}{Diferentes abordagens para caracterização de fontes sismogênicas}
			
			- \textbf{sismicidade}, 
			
			- geologia 
			
			- geodésia)
		\end{frame}

		\subsubsection{remoção de agrupamentos}
		\begin{frame}{Remoção de agrupamentos}
			- remoção de agrupamentos
		\end{frame}

		\subsubsection{magnitude de completude}
		\begin{frame}{Magnitude de completude}
			- magnitude de completude
		\end{frame}

		\subsubsection{magnitude máxima}
		\begin{frame}{Magnitude máxima}
			- magnitude máxima
		\end{frame}

		\subsubsection{relações de recorrência (MFD)}
		\begin{frame}{Relações de recorrência (MFD)}
			- relações de recorrência (MFD)
		\end{frame}


	\subsection{ii. magnitudes e rupturas associadas}
	\begin{frame}{Magnitudes e rupturas associadas}
		- magnitudes e rupturas associadas
	\end{frame}


	\subsection{iii. relações de atenuação (GMPEs)}
	\begin{frame}{Relações de atenuação (GMPEs)}
		- elações de atenuação (GMPEs)
	\end{frame}

	
	\subsection{iv. cálculo (propriamente dito) da ameaça.}
	\begin{frame}{Cálculo da ameaça}
	
	
	Cálculo da \textbf{curva de ameaça} para cada um dos locais (\emph{sites}), dados um
	conjunto de fontes sismicas e um conjunto de modelos de intensidade (GMPEs). 
	
	A probabilidade calculada é:
	
	\begin{equation}
	    \Pr(X \geq x ~|~ T) = 1 - \prod_i \prod_j \Pr_{rup_{ij}}(X < x ~|~ T)
	\end{equation}
    
    onde $\Pr(X \geq x | T)$ é a probabilidade de que o movimento do chão $X$
    exceda o nível $x$, uma ou mais vezes no período $T$ e $\Pr_{rup_{ij}}(X < x | T)$
 	é a probabilidade de que a $j$-ésima ruptura na $i$-ésima fonte sísmica não produza nenhuma
 	intensidade que exceda $x$ durante o período $T$. A primeira produtória é sobre as fontes
 	e a segunda sobre as rupturas na fonte.
 
	\\ ~
 
	Se computa a probabilidade certo nivel $x$ seja excedido ao menos uma vez,
	como 1 menos a probabilidade de que $x$ não seja excedido pelo movimento causado
	por nenhuma ruptura em nenhuma das fontes durante um mesmo intervalo de tempo $T$, com o 
	pressuposto de que as fontes e as rupturas nas fontes são \emph{independentes}.
		
	\end{frame}


% ====================================================
\section{10:40 - Pausa}
% ====================================================
	\begin{frame}{Pausa}
		- $H_2O$, 
		
		- Café, 
		
		- WCs,
		
		- \ldots
	\end{frame}

	\begin{frame}{Sugestão para apresentação em 5 minutos dos modelos de fontes}
		
		- apresentação geológica/tectônica da área de estudo (1 slide).
	
		- catálogo e zonas sísmicas (1 slide).
	
		- detalhamento de cada zona (1 slide por zona) sempre com ênfase nos métodos
		escolhidos/preferidos.
		
			\begin{itemize}
			  	\item pré-processamentos, se houveram;
				\item remoção de agrupamentos (decluster);
				\item magnitude mínima e de completude;
				\item número de eventos na zona
				\item magnitude máxima;
				\item Gutemberg-Richter/MFD/relações de recorrência:\\
				histograma com com taxa de sismicidade (valor-a) e valor-b ajustados.
			\end{itemize}
	
		- sismicidade de fundo (1 slide) como se fosse mais uma zona. 
	
	\end{frame}


% ====================================================
\section{10:50 - Modelos de fontes sismogênicas em construção no Brasil}
% ====================================================
	\subsection{Modelos de fontes sismogênicas em construção no Brasil}
	
	\begin{frame}{Modelos de fontes sismogênicas em construção no Brasil}
		
		- Brasil: Zonas. (Dourado) (5 min)
	
		- Brasil: Zonas. (Assumpção). (5 min)
		
		- Nordeste: Zonas. (Joaquim / Jordi). (5 min)
		
		- Sudeste: Zonas. (Tania / Jackeline). (5 min)
		
		- Zoneless / Smoothing. (Marlon). (5 min)
		
		- Breve discussão. (5 min)
	
	\end{frame}



% ====================================================
\section{11:20 - Árvore lógica para fontes sísmicas}
% ====================================================
	\subsection{Usar outras abordagens para definição de fontes ?!(10 min)}
	\begin{frame}{Usar outras abordagens ?!}
		- geológica
		
		- geodésica
	\end{frame}

	\subsection{Pesos para a arvore lógica (20 min)}
	\begin{frame}{Pesos para a arvore lógica}
		- Pesos para a arvore lógica
	\end{frame}


% ====================================================
\section{11:40 - Árvore lógica para GMPEs}
% ====================================================


	\subsection{Dados (de movimento forte) disponíveis (10 min)}
	\begin{frame}{Dados (de movimento forte) disponíveis}
		- Dados (de movimento forte) disponíveis.
		
		- Apresentação do Stéphane
	\end{frame}


	\subsection{Critérios de seleção das GMPEs (10 min)}
	\begin{frame}{Critérios de seleção das GMPEs}
		- GMPE's disponíveis para \emph{Stable Shalow Crust}.
	
		- critérios de seleção das GMPEs
		
		- Apresentação do Stéphane
	\end{frame}



% ====================================================
\section{12:00 - Compromissos e proposta de cronograma}
% ====================================================

	\begin{frame}{Principais compromissos e proposta de cronograma}
		\begin{itemize}
			\item Principais compromissos assumidos
			\item  Proposta de cronograma de execução
		\end{itemize}
	\end{frame}


	\begin{frame}{Muito obrigado a todos \\
				  pela presença e paciência}
		Onde sugerem o almoço ?!
	\end{frame}


% ====================================================
\section{14:00 - Detalhamento dos modelos de fontes sísmica (opcional)}
% ====================================================
	\begin{frame}{Definição e detalhamento do processamento das fontes-\emph{áreas}}
		\begin{itemize}
			\item definição e processamento de zonas sísmicas
			\item direções preferenciais de ruptura
			\item distribuição de profundidades
			\item falhamentos
		\end{itemize}
	\end{frame}
\end{document}

