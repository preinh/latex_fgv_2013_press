\documentclass[ucs,8pt]{beamer}

% Copyright 2004 by Till Tantau <tantau@users.sourceforge.net>.
%
% In principle, this file can be redistributed and/or modified under
% the terms of the GNU Public License, version 2.
%
% However, this file is supposed to be a template to be modified
% for your own needs. For this reason, if you use this file as a
% template and not specifically distribute it as part of a another
% package/program, I grant the extra permission to freely copy and
% modify this file as you see fit and even to delete this copyright
% notice.
%
% Modified by Tobias G. Pfeiffer <tobias.pfeiffer@math.fu-berlin.de>
% to show usage of some features specific to the FU Berlin template.

% remove this line and the "ucs" option to the documentclass when your editor is not utf8-capable
\usepackage[utf8]{inputenc}    % to make utf-8 input possible
\usepackage[brazil]{babel}
%\usepackage[english]{babel}     % hyphenation etc., alternatively use 'german' as parameter

% figure numbers, captions and subcaptions
\usepackage[font=small,format=plain,labelfont=bf,up,textfont=it,up,compatibility=false]{caption}
\usepackage[font=footnotesize,compatibility=false]{subcaption}
\usepackage[caption=false]{subfig}


\setbeamertemplate{caption}[numbered]

% links
%\usepackage[pdftex,plainpages=false,pdfpagelabels,pagebackref,colorlinks=true,citecolor=DarkGreen,linkcolor=NavyBlue,urlcolor=DarkRed,filecolor=Green,bookmarksopen=true]{hyperref}
\usepackage{hyperref}
\pdfcatalog{ /PageMode /FullScreen }

% bibliography
\usepackage[fixlanguage]{babelbib}
%\usepackage{biblatex}

% % links coloridos
% \usepackage[all]{hypcap}                    % soluciona o problema com o
% hyperref e capitulos
\usepackage[round,sort,nonamebreak]{natbib} % citação bibliográfica textual(plainnat-ime.bst)
\bibpunct{(}{)}{;}{a}{\hspace{-0.7ex},}{,} % estilo de citação. Veja alguns exemplos em http://merkel.zoneo.net/Latex/natbib.php


\usepackage[sanitize=none,acronym,toc]{glossaries}
%\usepackage[acronym,toc]{glossaries}
% Define a new glossary type
\newglossary[slg,toc]{symbols}{sym}{sbl}{Lista de Símbolos}
\newglossary[elg]{equations}{eqn}{eql}{Equations}
\makeglossaries


% ---------------------------------------------------------------------------- %
% Dicionario de Termos:
\input{d01-terms}       
\input{d02-symbols}       
\input{d03-acronyms}       
\input{d04-equations}       
% ---------------------------------------------------------------------------- %


\graphicspath{{./images/}}             % caminho das figuras (recomendável)
% Template for talks using the Corporate Design of the Freie Universitaet
%   Berlin, created following the guidelines on www.fu-berlin.de/cd by
%   Tobias G. Pfeiffer, <tobias.pfeiffer@math.fu-berlin.de>
% This file can be redistributed and/or modified in any way you like.
%   If you feel you have done significant improvements to this template,
%   please consider providing your modified version to
%   https://www.mi.fu-berlin.de/w/Mi/BeamerTemplateCorporateDesign

\usepackage{amsmath,dsfont,listings}

%%% FU logo
% small version for upper right corner of normal pages
\pgfdeclareimage[height=0.9cm]{university-logo}{images/logo_emap}
\logo{\pgfuseimage{university-logo}}
% large version for upper right corner of title page
\pgfdeclareimage[height=1.0cm]{big-university-logo}{images/logo_emap}
\newcommand{\titleimage}[1]{\pgfdeclareimage[height=6cm]{title-image}{#1}}
\titlegraphic{\pgfuseimage{title-image}}
%%% end FU logo

% NOTE: 1cm = 0.393 in = 28.346 pt;    1 pt = 1/72 in = 0.0352 cm
\setbeamersize{text margin right=3.5mm, text margin left=7.5mm}  % text margin

% colors to be used
\definecolor{text-grey}{rgb}{0.45, 0.45, 0.45} 	% grey text on white background
\definecolor{bg-grey}{rgb}{0.66, 0.65, 0.60} 	% grey background (for white text)
\definecolor{usp-blue}{RGB}{16, 148, 171} 			% blue text
\definecolor{usp-blue-light}{RGB}{100, 196, 208} 	% blue text
\definecolor{usp-orange}{RGB}{252, 180, 33} 		% Orange text
\definecolor{usp-red}{RGB}{204, 0, 0} 			% red text (used by \alert)

% switch off the sidebars
% TODO: loading \useoutertheme{sidebar} (which is maybe wanted) also inserts
%   a sidebar on title page (unwanted), also indents the page title (unwanted?),
%   and duplicates the navigation symbols (unwanted)
\setbeamersize{sidebar width left=0cm, sidebar width right=0mm}
\setbeamertemplate{sidebar right}{}
\setbeamertemplate{sidebar left}{}
%    XOR
% \useoutertheme{sidebar}

% frame title
% is truncated before logo and splits on two lines
% if neccessary (or manually using \\)
\setbeamertemplate{frametitle}{%
    \vskip-30pt \color{text-grey}\large%
    \begin{minipage}[b][23pt]{80.5mm}%
    \flushleft\insertframetitle%
    \end{minipage}%
}

%%% title page
% TODO: get rid of the navigation symbols on the title page.
%   actually, \frame[plain] *should* remove them...
\setbeamertemplate{title page}{
% upper right: FU logo
\vskip2pt\hfill\pgfuseimage{big-university-logo} \\
\vskip-20pt\hskip3pt
% title image of the presentation
\begin{minipage}{5.0cm}
\hspace{-.1cm}\inserttitlegraphic
\end{minipage}

% set the title and the author
\vskip-12pt
\parbox[top][1.35cm][c]{\textwidth}{\raggedleft{\color{text-grey}\inserttitle \\ \small \insertsubtitle}}
\vskip3pt
\parbox[top][1.35cm][c]{\textwidth}{\raggedleft{\small \insertauthor \\ \insertinstitute \\[3mm]
\scriptsize \insertdate}} }
%%% end title page

%%% colors
\usecolortheme{lily}
\setbeamercolor*{normal text}{fg=black,bg=white}
\setbeamercolor*{alerted text}{fg=usp-red}
\setbeamercolor*{example text}{fg=usp-orange}
\setbeamercolor*{structure}{fg=usp-blue}

\setbeamercolor*{block title}{fg=white,bg=black!50}
\setbeamercolor*{block title alerted}{fg=white,bg=black!50}
\setbeamercolor*{block title example}{fg=white,bg=black!50}

\setbeamercolor*{block body}{bg=black!10}
\setbeamercolor*{block body alerted}{bg=black!10}
\setbeamercolor*{block body example}{bg=black!10}

\setbeamercolor{bibliography entry author}{fg=usp-blue}
% TODO: this doesn't work at all:
\setbeamercolor{bibliography entry journal}{fg=text-grey}

\setbeamercolor{item}{fg=usp-blue}
\setbeamercolor{navigation symbols}{fg=text-grey,bg=bg-grey}
%%% end colors

%%% headline
\setbeamertemplate{headline}{
\vskip4pt\hfill\insertlogo\hspace{3.5mm} % logo on the right

\vskip6pt\color{usp-blue}\rule{\textwidth}{0.4pt} % horizontal line
}
%%% end headline

%%% footline
\newcommand{\footlinetext}{\insertshortinstitute, \insertshorttitle, \insertshortdate}
\setbeamertemplate{footline}{
\vskip5pt\color{usp-blue}\rule{\textwidth}{0.4pt}\\ % horizontal line
\vskip2pt
\makebox[123mm]{\hspace{7.5mm}
\color{usp-blue}\footlinetext
\hfill \raisebox{-1pt}{\usebeamertemplate***{navigation symbols}}
\hfill \insertframenumber}
\vskip4pt
}
%%% end footline

%%% settings for listings package
\lstset{extendedchars=true, showstringspaces=false, basicstyle=\footnotesize\sffamily, tabsize=2, breaklines=true, breakindent=10pt, frame=l, columns=fullflexible}
\lstset{language=Python} % this sets the syntax highlighting
\lstset{mathescape=true} % this switches on $...$ substitution in code
% enables UTF-8 in source code:
\lstset{literate={ä}{{\"a}}1 {ö}{{\"o}}1 {ü}{{\"u}}1 {Ä}{{\"A}}1 {Ö}{{\"O}}1 {Ü}{{\"U}}1 }
%%% end listings  % THIS is the line that includes the template!

\usepackage{arev,t1enc} % looks nicer than the standard sans-serif font
% if you experience problems, comment out the line above and change
% the documentclass option "9pt" to "10pt"


% ---------------------------------------------------------------------------- %
\DeclareMathOperator*{\argmin}{arg\,min}
\DeclareMathOperator*{\argmax}{arg\,max}
\DeclareMathOperator*{\erf}{erf}
% ---------------------------------------------------------------------------- %


% image to be shown on the title page (without file extension, should be pdf or png)

\titleimage{images/capa_pga}

\title[Smoothing Techniques] % (optional, use only with long paper titles)
{Técnicas de Suavização}

\subtitle
{aplicadas à caracterização de fontes sísmicas e \\ 
 à análise probabilística de ameaça sísmica (PSHA)}

\author[Pirchiner, Marlon] % (optional, use only with lots of authors)
{M.~Pirchiner }
% - Give the names in the same order as the appear in the paper.

\institute[EMAp-FGV / IAG-USP] % (optional, but mostly needed)
{EMAp-FGV / Centro de Sismologia - USP}
% - Keep it simple, no one is interested in your street address.

\date[EMAP2014] % (optional, should be abbreviation of conference name)
{Defesa de Mestrado, julho de 2014}
% - Either use conference name or its abbreviation.
% - Not really informative to the audience, more for people (including
%   yourself) who are reading the slides online

\subject{Seismology, Earthquake, Seismic Hazard, Smoothing}
% This is only inserted into the PDF information catalog. Can be left
% out.

% you can redefine the text shown in the footline. use a combination of
% \insertshortauthor, \insertshortinstitute, \insertshorttitle, \insertshortdate, ...
\renewcommand{\footlinetext}{\insertshortinstitute, \insertshorttitle, \insertshortdate}

% Delete this, if you do not want the table of contents to pop up at
% the beginning of each subsection:
\AtBeginSubsection{
  \begin{frame}<beamer>{Agenda}
    \tableofcontents[currentsection, currentsubsection]
  \end{frame}
  }


\begin{document}



\begin{frame}[plain]
  \titlepage
\end{frame}

\begin{frame}{Agenda}
  \tableofcontents
  % You might wish to add the option [pausesections]
\end{frame}

%-----------------------------------------------------------------------------
\begin{frame}{slide}

\begin{figure}[H]
   \centering
   \includegraphics[height=0.95\textheight]{global_pde_mag_all}
   \caption[Mapa Mundial de Epicentros 1963-1998]
   		   {Mapa Mundial de Epicentros 1963-1998 - \citet{lowman_jr_1998}} 
   \label{f:global_epicenters}
\end{figure} 


\end{frame}


%-----------------------------------------------------------------------------
\begin{frame}{slide}

\begin{figure}[H]
   \centering
   \includegraphics[height=0.95\textheight]{litho_plates_overview}
   \caption[Cartografia das placas litosféricas]
   		   {Cartografia das placas litosféricas - \citet{usgs_plates_1996}} 
   \label{f:plates_overview}
\end{figure} 

\end{frame}




%-----------------------------------------------------------------------------
\begin{frame}{slide}

\begin{figure}[H]
   \centering
   \includegraphics[height=0.95\textheight]{mfd}
   \caption[Distribuições de frequência e magnitude]
   		   {Distribuições de frequência e magnitude} 
   \label{f:mfd}
\end{figure} 

\end{frame}



%-----------------------------------------------------------------------------
\begin{frame}{slide}

\begin{figure}[H]
   \centering
   \includegraphics[height=0.95\textheight]{occurrence}
   \caption[Distribuição incremental e cumulativa de frequência e magnitude dos sismos presentes no catálogo ISC-GEM
   para a América do Sul unido com o \gls{bsb2013}]
   {Distribuição incremental e cumulativa de frequência e magnitude dos sismos presentes no catálogo ISC-GEM
   para a América do Sul unido com o \gls{bsb2013}} 
   \label{f:occurrence}
\end{figure} 

\end{frame}



%-----------------------------------------------------------------------------
\begin{frame}{slide}

\begin{figure}[H]
  \centering
  \includegraphics[height=.70\textheight]{seismicity_sa} 
  \caption{Sismicidade da América do Sul, Catálogo ISC-GEM (seção \ref{sec:data_source}) . 
  			A geologia ao fundo é fonte da \gls{cgmw} via OneGeology.
		   Sismos mais profundos foram registrados no interior da placa, inclusive sobre o Acre.
  		   }
  \label{fig:sa_seis} 
\end{figure}


\end{frame}



%-----------------------------------------------------------------------------
\begin{frame}{slide}


\begin{figure}[H]
  \centering
  \includegraphics[height=.70\textheight]{tectonico_brasil} 
  \caption{Mapa Geológico do Brasil\footnotemark em escala 1:1.000.000}.
  \label{fig:br_tec} 
\end{figure}
\footnotetext{\citet{bizzi_2003}}


\end{frame}



%-----------------------------------------------------------------------------
\begin{frame}{slide}


\begin{figure}[H]
  \centering
  \includegraphics[height=.90\textheight]{seismicity_br} 
  \caption{Sismicidade do Brasil. Catálogo \gls{bsb2013} (seção \ref{sec:data_source2}).}
  \label{fig:br_seis} 
\end{figure}


\end{frame}



%-----------------------------------------------------------------------------
\begin{frame}{slide}


\begin{figure}[H]
	\centering
	\begin{subfigure}[b]{0.48\textheight}
		  	\centering
			\includegraphics[height=1.00\textheight]{hmtk_sa3_rate}
			\subcaption{Número de tremores registrados por ano, \gls{iscgem}}
			\label{fig:sa_eq_record}
	\end{subfigure}%
	\quad %~ %add desired spacing between images, e. g. ~, \quad, \qquad, \hfill etc.
	\begin{subfigure}[b]{0.48\textheight}
		  	\centering
			\includegraphics[height=1.00\textheight]{hmtk_bsb2013_rate}
			\subcaption{Número de tremores registrados por ano, \gls{bsb2013}}
			\label{fig:br_eq_record}
    \end{subfigure}%
	\caption{Número de tremores registrados por ano após 1900}
	\label{fig:eq_record}
\end{figure}




\end{frame}



%-----------------------------------------------------------------------------
\begin{frame}{slide}

\begin{figure}[!h]
  \centering
  \includegraphics[height=.80\textheight]{oq_ecosystem} 
  \caption{Ecossistema de módulos, bibliotecas e utilitários do OpenQuake}
  \label{fig:oq} 
\end{figure}



\end{frame}



%-----------------------------------------------------------------------------
\begin{frame}{slide}
\begin{figure}[H]
	\centering
	\begin{subfigure}[b]{0.45\textheight}
		  	\centering
			\includegraphics[height=1.00\textheight]{hmtk_sa3_weekday}
			\subcaption{Distribuição dos tremores nos dias da semana, \gls{iscgem}}
			\label{fig:sa_week_hist}
	\end{subfigure}%
	\quad %~ %add desired spacing between images, e. g. ~, \quad, \qquad, \hfill etc.
	\begin{subfigure}[b]{0.45\textheight}
		  	\centering
			\includegraphics[height=1.00\textheight]{hmtk_bsb2013_weekday}
			\subcaption{Distribuição dos tremores nos dias da semana, \gls{bsb2013}}
			\label{fig:br_week_hist}
    \end{subfigure}%
        %~ %add desired spacing between images, e. g. ~, \quad, \qquad, \hfill etc.
          %(or a blank line to force the subfigure onto a new line)
          
 	\begin{subfigure}[b]{0.45\textheight}
		  	\centering
			\includegraphics[height=1.00\textheight]{hmtk_sa3_hour}
			\subcaption{Distribuição do horário de ocorrência dos tremores, \gls{iscgem}}
			\label{fig:sa_hour_hist}
	\end{subfigure}%
	\quad %~ %add desired spacing between images, e. g. ~, \quad, \qquad, \hfill etc.
	\begin{subfigure}[b]{0.45\textheight}
		  	\centering
			\includegraphics[height=1.00\textheight]{hmtk_bsb2013_hour}
			\subcaption{Distribuição do horário de ocorrência dos tremores, \gls{bsb2013}}
			\label{fig:br_hour_hist}
    \end{subfigure}%
        %~ %add desired spacing between images, e. g. ~, \quad, \qquad, \hfill etc.
          %(or a blank line to force the subfigure onto a new line)

	\begin{subfigure}[b]{0.45\textheight}
		  	\centering
			\includegraphics[height=1.00\textheight]{dep_sa_hist}
			\subcaption{Distribuição da profundidade dos tremores, \gls{iscgem}}
			\label{fig:sa_dep_hist}
	\end{subfigure}%
	\quad %~ %add desired spacing between images, e. g. ~, \quad, \qquad, \hfill etc.
	\begin{subfigure}[b]{0.4\textheight}
		  	\centering
			\includegraphics[height=1.00\textheight]{dep_br_hist}
			\subcaption{Distribuição da profundidade dos tremores, \gls{bsb2013}}
			\label{fig:br_dep_hist}
        \end{subfigure}%
        %~ %add desired spacing between images, e. g. ~, \quad, \qquad, \hfill etc.

  \caption{Checagem de qualidade.}
  \label{fig:qc_histograms} 
\end{figure}



\end{frame}



%-----------------------------------------------------------------------------
\begin{frame}{slide}

\begin{figure}[H]
	\centering
	\begin{subfigure}[b]{0.48\textheight}
		  	\centering
			\includegraphics[height=1.00\textheight]{decluster_sa}
			\subcaption{Número cumulativo de tremores registrados por ano para o \gls{iscgem}
			original e para diferentes métodos/janelas de remoção de agrupamentos.}
			\label{fig:sa_eq_record}
	\end{subfigure}%
	\quad %~ %add desired spacing between images, e. g. ~, \quad, \qquad, \hfill etc.
	\begin{subfigure}[b]{0.48\textheight}
		  	\centering
			\includegraphics[height=1.00\textheight]{decluster_br}
			\subcaption{Número cumulativo de tremores registrados por ano para o \gls{bsb2013}
			original e para diferentes métodos/janelas de remoção de agrupamentos.}
			\label{fig:br_eq_record}
    \end{subfigure}%
	\caption{Número cumulativo de tremores registrados por ano após 1900}
	\label{fig:eq_decluster_cum}
\end{figure}



\end{frame}



%-----------------------------------------------------------------------------
\begin{frame}{slide}

\begin{figure}[H]
	\centering
	\begin{subfigure}[t]{0.46\textheight}
		  	\centering
			\includegraphics[height=1.00\textheight]{hmtk_sa3_pp_decluster}
			\subcaption{Número de tremores registrados por ano, \gls{iscgem}}
			\label{fig:sa_decluster}
	\end{subfigure}%
	\quad %~ %add desired spacing between images, e. g. ~, \quad, \qquad, \hfill etc.
	\begin{subfigure}[t]{0.50\textheight}
		  	\centering
			\includegraphics[height=1.00\textheight]{hmtk_bsb2013_pp_decluster}
			\subcaption{Número de tremores registrados por ano, \gls{bsb2013}}
			\label{fig:br_decluster}
    \end{subfigure}%
	\caption{Número de tremores registrados por ano após 1900}
	\label{fig:eq_decluster}
\end{figure}

\end{frame}



%-----------------------------------------------------------------------------
\begin{frame}{slide}

\begin{figure}[H]
	  \centering
	  \begin{subfigure}[b]{0.7\textheight}
		  	\centering
			\includegraphics[height=1.00\textheight]{time_mag_count_sa}
			\caption{Catálogo \gls{iscgem} 1900-2012}
			\label{fig:tmf_sa}
        \end{subfigure}%
        %~ %add desired spacing between images, e. g. ~, \quad, \qquad, \hfill etc.
          %(or a blank line to force the subfigure onto a new line)

	  \begin{subfigure}[b]{0.7\textheight}
		  	\centering
  			\includegraphics[height=1.00\textheight]{time_mag_count_br}
			\caption{Catálogo \gls{bsb2013} 1900-2012}
			\label{fig:tmf_br}
       \end{subfigure}%

	   \begin{subfigure}[b]{0.7\textheight}
		  	\centering
  			\includegraphics[height=1.00\textheight]{time_mag_count_br_1960}
			\caption{Catálogo \gls{bsb2013} a partir de 1960-2012}
			\label{fig:tmf_br_1960}
       \end{subfigure}%

  \caption{Contagem de sismos em tempo e em magnitude.}
  \label{fig:qc_time_mag_count} 
\end{figure}



\end{frame}



%-----------------------------------------------------------------------------
\begin{frame}{slide}

\begin{figure}[H]
	\centering
	\begin{subfigure}[b]{0.47\textheight}
		  	\centering
			\includegraphics[height=1.00\textheight]{stepp_sa}
			\subcaption{Diagrama de Stepp para o \gls{iscgem} (\emph{declustered})}
			\label{fig:sa_stepp}
	\end{subfigure}%
	\quad %~ %add desired spacing between images, e. g. ~, \quad, \qquad, \hfill etc.
	\begin{subfigure}[b]{0.47\textheight}
		  	\centering
			\includegraphics[height=1.00\textheight]{stepp_br}
			\subcaption{Diagrama de Stepp para o \gls{bsb2013} (\emph{declustered})}
			\label{fig:br_stepp}
    \end{subfigure}%
	\caption{Diagrama de Stepp para análise da magnitude de completude \gls{sym:Mc}}
	\label{fig:eq_stepp}
\end{figure}


\end{frame}



%-----------------------------------------------------------------------------
\begin{frame}{slide}

	\begin{table}[h]
	  	\centering
		\begin{tabular}{l|*{11}{c}}
		$M_c$ & 3.0  & 3.5  & 4.0  & 4.5  & 5.0  & 5.5  & 6.0  & 6.5  & 7.0  & 7.5  & 8 \\  \hline
		Ano   & 1986 & 1986 & 1986 & 1960 & 1958 & 1958 & 1927 & 1898 & 1885 & 1885 & 1885   \\
		\end{tabular}
		\caption{Magnitude de completude, \gls{iscgem}}
		\label{tab:mc_sa}
	\end{table}

	\begin{table}[h]
	  	\centering
		\begin{tabular}{l|*{7}{c}}
		$M_c$ & 3.0  & 3.5  & 4.0  & 4.5  & 5.0  & 5.5  & 6.0  \\  \hline
		Ano   & 1980 & 1975 & 1975 & 1965 & 1965 & 1860 & 1860 \\
		\end{tabular}
		\caption{Magnitude de completude, Cat.BSB2013}
		\label{tab:mc_br}
	\end{table}

\end{frame}



%-----------------------------------------------------------------------------
\begin{frame}{slide}


\begin{figure}[H]
  \centering
  \includegraphics[height=.80\textheight]{woo_bandwidth} 
  \caption{Ajuste da largura de banda para o método de Woo1996}
  \label{fig:woo_b} 
\end{figure}


\end{frame}



%-----------------------------------------------------------------------------
\begin{frame}{slide}


\begin{figure}[H]
  \centering
  \includegraphics[height=.80\textheight]{helmstetter_catalogues} 
  \caption{Catálogos de aprendizado e de teste para o método de \citet{helmstetter_2012}}
  \label{fig:h_catalogue} 
\end{figure}


\end{frame}



%-----------------------------------------------------------------------------
\begin{frame}{slide}


\begin{figure}[H]
  \centering
  \includegraphics[height=.80\textheight]{helmstetter_hidi} 
  \caption{Exemplo da largura de banda para um determinado evento para o método de Helmstetter, com $k_{cnn} = 5$ e
  $a_{cnn} = 100$}
  \label{fig:h_hidi} 
\end{figure}


\end{frame}



%-----------------------------------------------------------------------------
\begin{frame}{slide}

\begin{figure}[H]
  \centering
  \includegraphics[height=.80\textheight]{helmstetter_stationary_a} 
  \caption{Taxa de sismicidade estacionaria calculara a partir da mediana da taxa de sismicidade
  modelada pelo método de Helmstetter2012 para uma determinada célula $r_0$}
  \label{fig:h_stationary} 
\end{figure}

\end{frame}



%-----------------------------------------------------------------------------
\begin{frame}{slide}

\begin{figure}[H]
	\centering
	\begin{tabular}{l}
	\includegraphics[height=0.80\textheight]{classical_psha_workflow}
	\end{tabular}
	\caption{Fluxo de trabalho clássico para a \gls{psha} \citep{crowley_2013}.}
\label{fig:classical_psha}
\end{figure}


\end{frame}



%-----------------------------------------------------------------------------
\begin{frame}{slide}

\begin{figure}[H]
  \centering
  \includegraphics[height=.80\textheight]{pga_gshap} 
  \caption{Resultado do GSHAP para o Brasil: \gls{PGA} (10\%/50anos) em unidades de $g$}
  \label{fig:gshap} 
\end{figure}

\end{frame}



%-----------------------------------------------------------------------------
\begin{frame}{slide}

\begin{figure}[H]
  \centering
  \includegraphics[height=.80\textheight]{a_dourado} 
  \caption{Zoneamento sísmico e caracterização das zonas sísmicas por \citep{dourado_2014}.
  Os valores para a magnitude mínima foram de 3.0}
  \label{fig:a_dourado} 
\end{figure}



\end{frame}



%-----------------------------------------------------------------------------
\begin{frame}{slide}

\begin{figure}[H]
  \centering
  \includegraphics[height=.80\textheight]{pga_dourado_oq} 
  \caption{Mapa de ameaça sísmica, PGA(poe 0.1, 50y)[Dourado, 20014] OpenQuake-Engine }
  \label{fig:pga_dourado_oq} 
\end{figure}

\end{frame}



%-----------------------------------------------------------------------------
\begin{frame}{slide}

\begin{figure}[H]
  \centering
  \includegraphics[height=.80\textheight]{a_frankel_br} 
  \caption{Mapa do valor-a, usando o catálogo \gls{bsb2013} calculado pelo método de Frankel, 1995 }
  \label{fig:a_fran_br} 
\end{figure}

\end{frame}



%-----------------------------------------------------------------------------
\begin{frame}{slide}

\begin{figure}[H]
  \centering
  \includegraphics[height=.80\textheight]{a_frankel_sa} 
  \caption{Mapa do valor-a, catálogo \gls{iscgem} [Frankel, 1995] }
  \label{fig:a_fran_sa} 
\end{figure}

\end{frame}



%-----------------------------------------------------------------------------
\begin{frame}{slide}

\begin{figure}[H]
  \centering
  \includegraphics[height=.80\textheight]{pga_frankel} 
  \caption{Mapa de ameaça sísmica, PGA (poe 10\%, 50y) [Frankel, 1995] }
  \label{fig:pga_fran} 
\end{figure}

\end{frame}



%-----------------------------------------------------------------------------
\begin{frame}{slide}

\begin{figure}[H]
  \centering
  \includegraphics[height=.80\textheight]{a_woo} 
  \caption{Mapa do valor-a, usando o catálogo \gls{bsb2013} calculado pelo método de Woo, 1996 }
  \label{fig:a_woo} 
\end{figure}

\end{frame}



%-----------------------------------------------------------------------------
\begin{frame}{slide}

\begin{figure}[H]
  \centering
  \includegraphics[height=.80\textheight]{pga_woo_inc} 
  \caption{Mapa de ameaça sísmica, PGA (poe 10\%, 50y) 
  		   calculado com o Openquake a partir das fontes sísmicas
  		   determinas pelo método de Woo, 1996, usando uma \gls{mfd}
  		   discreta e incremental.
  }
  \label{fig:pga_woo_inc} 
\end{figure}

\end{frame}



%-----------------------------------------------------------------------------
\begin{frame}{slide}

\begin{figure}[H]
	\centering
	\begin{subfigure}[t]{0.47\textheight}
		\centering
		\includegraphics[height=1.0\textheight]{pga_woo_cum} 
		\subcaption{Mapa de ameaça sísmica, PGA (poe 10\%, 50y), 
  		   calculado com o Openquake a partir das fontes sísmicas
  		   determinas pelo método de Woo, 1996, usando uma \gls{mfd}
  		   truncada usando o valor-a como o calor cumulativo
  		   contado a partir da \gls{sym:Mmin}.
		}
		\label{fig:pga_woo_cum} 
	\end{subfigure}
	\quad
	\begin{subfigure}[t]{0.47\textheight}
		\centering
		\includegraphics[height=1.0\textheight]{pga_woo_dif} 
		\subcaption{Mapa diferencial de ameaça, PGA (poe 10\%, 50y), 
		   entre os modelos usando \gls{mfd} truncada e \gls{mfd}
		   discreta. Diferença entre os mapas das figuras 
		   \ref{fig:pga_woo_inc} e \ref{fig:pga_woo_cum}.
		   }
		\label{fig:pga_woo_dif} 
	\end{subfigure}
	\caption{Variação do resultado da ameaça em função do uso de diferentes 
			\glspl{mfd} no OpenQuake}
	\label{fig:pga_woo} 
\end{figure}


\end{frame}



%-----------------------------------------------------------------------------
\begin{frame}{slide}

\begin{table}[H]
	\centering
	\begin{tabular}{c|c}
		Parâmetro & Valor \\ \hline
		$R_{min}$ & $0.1\times10^{-13}$ \\
		$a_{cnn}$ & 325 \\
		$k_{cnn}$ & 1 \\
	\end{tabular}
	\caption{Parâmetros otimizados para o modelo de Helmstetter a partir do catálogo \gls{bsb2013}}
	\label{tab:hemlstetter}
\end{table}

\end{frame}



%-----------------------------------------------------------------------------
\begin{frame}{slide}

\begin{figure}[H]
  \centering
  \includegraphics[height=.80\textheight]{a_helmstetter} 
  \caption{Mapa do valor-a, usando o catálogo \gls{bsb2013} calculado pelo método de Helmstetter, 2012 }
  \label{fig:helm_r} 
\end{figure}

\end{frame}



%-----------------------------------------------------------------------------
\begin{frame}{slide}

\begin{figure}[H]
  \centering
  \includegraphics[height=.80\textheight]{pga_helmstetter} 
  \caption{Mapa de ameaça sísmica, PGA (poe 10\%, 50y), 
  		   calculado com o OpenQuake a partir das fontes sísmicas
  		   determinas pelo método de Helmstetter,2012 }
  \label{fig:helm_h} 
\end{figure}

\end{frame}



%-----------------------------------------------------------------------------
\begin{frame}{slide}


\end{frame}



%-----------------------------------------------------------------------------
\begin{frame}{slide}


\end{frame}



%-----------------------------------------------------------------------------
\begin{frame}{slide}


\end{frame}



%-----------------------------------------------------------------------------
\begin{frame}{slide}


\end{frame}



%-----------------------------------------------------------------------------
\begin{frame}{slide}


\end{frame}



%-----------------------------------------------------------------------------
\begin{frame}{slide}


\end{frame}



%-----------------------------------------------------------------------------
\begin{frame}{slide}


\end{frame}



%-----------------------------------------------------------------------------
\begin{frame}{slide}


\end{frame}



%-----------------------------------------------------------------------------
\begin{frame}{slide}


\end{frame}



%-----------------------------------------------------------------------------
\begin{frame}{slide}


\end{frame}



%-----------------------------------------------------------------------------
\begin{frame}{slide}


\end{frame}




\begin{frame}{Make Titles Informative. Use Uppercase Letters. Long Titles are Split Automatically.}{Subtitles are optional.}
  % - A title should summarize the slide in an understandable fashion
  %   for anyone how does not follow everything on the slide itself.

	\begin{figure}[H]
	  \centering
	  \includegraphics[height=.90\textheight]{seismicity_br} 
	  \caption{Sismicidade do Brasil. Catálogo \gls{bsb2013}.}
	  \label{fig:br_seis} 
	\end{figure}


\end{frame}


\begin{frame}{Figura teste}
explicar algum item por exemplo
	\begin{itemize}
		\item algum argumento pra dar
	\end{itemize}
	\begin{figure}[H]
		\centering{}
		\includegraphics[height=0.70\textheight]{global_pde_mag_all}
		\caption[Mapa Mundial de Epicentros 1963-1998]
		{Mapa Mundial de Epicentros 1963-1998 \citet{lowman_jr_1998}}
		\label{f:global_epicenters}
\end{figure}
\end{frame}



%
\begin{frame}{Figura teste}
explicar algum item por exemplo
\begin{itemize}
	\item algum argumento pra dar
\end{itemize}

\begin{figure}[H]
   \centering
   \includegraphics[height=0.70\textheight]{global_pde_mag_all}
   \caption[Mapa Mundial de Epicentros 1963-1998]
   		   {Mapa Mundial de Epicentros 1963-1998 \citet{lowman_jr_1998} } 
   \label{f:global_epicenters}
\end{figure}
\end{frame}


\begin{frame}{Subfiguras outro teste}
se eu disser algo por aqui dá zica \\
testa pra ver
\begin{figure}
	\centering
	\begin{subfigure}[t]{0.48\textheight}
		  	\centering
			\includegraphics[height=1.00\textheight]{hmtk_sa3_rate}
			\subcaption{Número de tremores registrados por ano, \gls{iscgem}}
			\label{fig:sa_eq_record}
	\end{subfigure}%
	\quad %~ %add desired spacing between images, e. g. ~, \quad, \qquad, \hfill etc.
	\begin{subfigure}[t]{0.48\textheight}
		  	\centering
			\includegraphics[height=1.00\textheight]{hmtk_bsb2013_rate}
			\subcaption{Número de tremores registrados por ano, \gls{bsb2013}}
			\label{fig:br_eq_record}
    \end{subfigure}
	\caption{Número de tremores registrados por ano após 1900}
	\label{fig:eq_record}
\end{figure}
\end{frame}



\begin{frame}{Make Titles Informative.}

  You can create overlays\dots
  \begin{itemize}
  \item using the \texttt{pause} command:
    \begin{itemize}
    \item
      First item.
      \pause
    \item    
      Second item.
    \end{itemize}
  \item
    using overlay specifications:
    \begin{itemize}
    \item<3->
      First item.
    \item<4->
      Second item.
    \end{itemize}
  \item
    using the general \texttt{uncover} command:
    \begin{itemize}
      \uncover<5->{\item
        First item.}
      \uncover<6->{\item
        Second item.}
    \end{itemize}
  \end{itemize}
\end{frame}



\begin{frame}{Subfiguras outro teste}
\begin{figure}
	\centering
	\begin{subfigure}[b]{0.48\textheight}
		  	\centering
			\includegraphics[height=1.00\textheight]{hmtk_sa3_rate}
			\subcaption{Número de tremores registrados por ano, \gls{iscgem}}
			\label{fig:sa_eq_record}
	\end{subfigure}%
	\quad %~ %add desired spacing between images, e. g. ~, \quad, \qquad, \hfill etc.
	\begin{subfigure}[b]{0.48\textheight}
		  	\centering
			\includegraphics[height=1.00\textheight]{hmtk_bsb2013_rate}
			\subcaption{Número de tremores registrados por ano, \gls{bsb2013}}
			\label{fig:br_eq_record}
    \end{subfigure}
	\caption{Número de tremores registrados por ano após 1900}
	\label{fig:eq_record}
\end{figure}
\end{frame}

\subsection{Previous Work}

\begin{frame}[fragile]{An old algorithm}
% NB. listings is quite powerful, but not well suited to be used with beamer
%  consider using semiverbatim or the like, see below
\begin{lstlisting}[language=C]
int main (void)
{
  std::vector<bool> is_prime (100, true);
  for (int i = 2; i < 100; i++)
    if (is_prime[i])
      {
        std::cout << i << " ";
        for (int j = i; j < 100;
            is_prime [j] = false, j+=i);
      }
  return 0;
}
\end{lstlisting}
\end{frame}

\begin{frame}[fragile]
  \frametitle{An Algorithm For Finding Primes Numbers.}
\begin{semiverbatim}
\uncover<1->{\alert<0>{int main (void)}}
\uncover<1->{\alert<0>{\{}}
\uncover<1->{\alert<1>{ \alert<4>{std::}vector<bool> is_prime (100, true);}}
\uncover<1->{\alert<1>{ for (int i = 2; i < 100; i++)}}
\uncover<2->{\alert<2>{    if (is_prime[i])}}
\uncover<2->{\alert<0>{      \{}}
\uncover<3->{\alert<3>{        \alert<4>{std::}cout << i << " ";}}
\uncover<3->{\alert<3>{        for (int j = i; j < 100;}}
\uncover<3->{\alert<3>{             is_prime [j] = false, j+=i);}}
\uncover<2->{\alert<0>{      \}}}
\uncover<1->{\alert<0>{ return 0;}}
\uncover<1->{\alert<0>{\}}}
\end{semiverbatim}
  \visible<4->{Note the use of \alert{\texttt{std::}}.}
\end{frame}

\section{Our Results/Contribution}

\subsection{Main Results}

\begin{frame}{Make Titles Informative.}
  \begin{example}
    \begin{itemize}
    \item 2 is prime (two divisors: 1 and 2).
    \item 3 is prime (two divisors: 1 and 3).
    \item 4 is not prime (\alert{three} divisors: 1, 2, and 4).
    \end{itemize}
  \end{example}
\end{frame}

\begin{frame}{Make Titles Informative.}
\begin{theorem}
 There is no largest prime number and, in addition, $$\int_\Omega \nabla u \cdot \nabla v = - \int_\Omega u \Delta v + \int_{\partial\Omega} u v n$$
 \end{theorem}
 \begin{proof}
 \begin{enumerate}
 \item<1-> Suppose $p$ were the largest prime number.
 \item<2-> Let $q$ be the product of the first $p$ numbers.
 \item<3-> Then $q + 1$ is not divisible by any of them.
 \item<1-> Thus $q + 1$ is also prime and greater than $p$.\qedhere
 \end{enumerate} 
 \end{proof}
 \uncover<4->{The proof used \textit{reductio ad absurdum}.}
\end{frame}

\begin{frame}{Make Titles Informative.}
\end{frame}


\subsection{Basic Ideas for Proofs/Implementation}

\begin{frame}{Make Titles Informative.}
\end{frame}

\begin{frame}{Make Titles Informative.}
\end{frame}

\begin{frame}{Make Titles Informative.}
\end{frame}



\section*{Summary}

\begin{frame}{Summary}

  % Keep the summary *very short*.
  \begin{itemize}
  \item
    The \alert{first main message} of your talk in one or two lines.
  \item
    The \alert{second main message} of your talk in one or two lines.
  \item
    Perhaps a \alert{third message}, but not more than that.
  \end{itemize}
  
  % The following outlook is optional.
  \vskip0pt plus.5fill
  \begin{itemize}
  \item
    Outlook
    \begin{itemize}
    \item
      Something you haven't solved.
    \item
      Something else you haven't solved.
    \end{itemize}
  \end{itemize}
\end{frame}



% All of the following is optional and typically not needed. 
\appendix
\section<presentation>*{\appendixname}
\subsection<presentation>*{Referências}



\begin{frame}[allowframebreaks]{Referências}
	\scriptsize
	% ---------------------------------------------------------------------------- %
	% Bibliografia
	\backmatter \singlespacing   				% espaçamento simples
	%\bibliographystyle{plain} 	% citação bibliográfica textual
	\bibliographystyle{styles/plainnat-ime} 	% citação bibliográfica textual
	\bibliography{bib/bibliografia}  			% associado ao arquivo: 'bibliografia.bib'

\end{frame}


\end{document}

