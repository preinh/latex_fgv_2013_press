\documentclass[ucs,8pt]{beamer}

% Copyright 2004 by Till Tantau <tantau@users.sourceforge.net>.
%
% In principle, this file can be redistributed and/or modified under
% the terms of the GNU Public License, version 2.
%
% However, this file is supposed to be a template to be modified
% for your own needs. For this reason, if you use this file as a
% template and not specifically distribute it as part of a another
% package/program, I grant the extra permission to freely copy and
% modify this file as you see fit and even to delete this copyright
% notice.
%
% Modified by Tobias G. Pfeiffer <tobias.pfeiffer@math.fu-berlin.de>
% to show usage of some features specific to the FU Berlin template.

% remove this line and the "ucs" option to the documentclass when your editor is not utf8-capable
\usepackage[utf8]{inputenc}    % to make utf-8 input possible
\usepackage[brazil]{babel}
%\usepackage[english]{babel}     % hyphenation etc., alternatively use 'german' as parameter

% figure numbers, captions and subcaptions
\usepackage[font=footnotesize,format=plain,labelfont=bf,up,textfont=it,up,compatibility=false]{caption}
\usepackage[font=scriptsize,compatibility=false]{subcaption}
\usepackage[caption=false]{subfig}


\setbeamertemplate{caption}[numbered]

% links
%\usepackage[pdftex,plainpages=false,pdfpagelabels,pagebackref,colorlinks=true,citecolor=DarkGreen,linkcolor=NavyBlue,urlcolor=DarkRed,filecolor=Green,bookmarksopen=true]{hyperref}
\usepackage{hyperref}
%\pdfcatalog{ /PageMode /FullScreen }

% bibliography
\usepackage[fixlanguage]{babelbib}
%\usepackage{biblatex}

% % links coloridos
% \usepackage[all]{hypcap}                    % soluciona o problema com o
% hyperref e capitulos
\usepackage[round,sort,nonamebreak]{natbib} % citação bibliográfica textual(plainnat-ime.bst)
\bibpunct{(}{)}{;}{a}{\hspace{-0.7ex},}{,} % estilo de citação. Veja alguns exemplos em http://merkel.zoneo.net/Latex/natbib.php


\usepackage[sanitize=none,acronym,toc]{glossaries}
%\usepackage[acronym,toc]{glossaries}
% Define a new glossary type
\newglossary[slg,toc]{symbols}{sym}{sbl}{Lista de Símbolos}
\newglossary[elg]{equations}{eqn}{eql}{Equations}
\makeglossaries


% ---------------------------------------------------------------------------- %
% Dicionario de Termos:
\input{d01-terms}       
\input{d02-symbols}       
\input{d03-acronyms}       
\input{d04-equations}       
% ---------------------------------------------------------------------------- %


\graphicspath{{./images/}}             % caminho das figuras (recomendável)
% Template for talks using the Corporate Design of the Freie Universitaet
%   Berlin, created following the guidelines on www.fu-berlin.de/cd by
%   Tobias G. Pfeiffer, <tobias.pfeiffer@math.fu-berlin.de>
% This file can be redistributed and/or modified in any way you like.
%   If you feel you have done significant improvements to this template,
%   please consider providing your modified version to
%   https://www.mi.fu-berlin.de/w/Mi/BeamerTemplateCorporateDesign

\usepackage{amsmath,dsfont,listings}

%%% FU logo
% small version for upper right corner of normal pages
\pgfdeclareimage[height=0.9cm]{university-logo}{images/logo_emap}
\logo{\pgfuseimage{university-logo}}
% large version for upper right corner of title page
\pgfdeclareimage[height=1.0cm]{big-university-logo}{images/logo_emap}
\newcommand{\titleimage}[1]{\pgfdeclareimage[height=6cm]{title-image}{#1}}
\titlegraphic{\pgfuseimage{title-image}}
%%% end FU logo

% NOTE: 1cm = 0.393 in = 28.346 pt;    1 pt = 1/72 in = 0.0352 cm
\setbeamersize{text margin right=3.5mm, text margin left=7.5mm}  % text margin

% colors to be used
\definecolor{text-grey}{rgb}{0.45, 0.45, 0.45} 	% grey text on white background
\definecolor{bg-grey}{rgb}{0.66, 0.65, 0.60} 	% grey background (for white text)
\definecolor{usp-blue}{RGB}{16, 148, 171} 			% blue text
\definecolor{usp-blue-light}{RGB}{100, 196, 208} 	% blue text
\definecolor{usp-orange}{RGB}{252, 180, 33} 		% Orange text
\definecolor{usp-red}{RGB}{204, 0, 0} 			% red text (used by \alert)

% switch off the sidebars
% TODO: loading \useoutertheme{sidebar} (which is maybe wanted) also inserts
%   a sidebar on title page (unwanted), also indents the page title (unwanted?),
%   and duplicates the navigation symbols (unwanted)
\setbeamersize{sidebar width left=0cm, sidebar width right=0mm}
\setbeamertemplate{sidebar right}{}
\setbeamertemplate{sidebar left}{}
%    XOR
% \useoutertheme{sidebar}

% frame title
% is truncated before logo and splits on two lines
% if neccessary (or manually using \\)
\setbeamertemplate{frametitle}{%
    \vskip-30pt \color{text-grey}\large%
    \begin{minipage}[b][23pt]{80.5mm}%
    \flushleft\insertframetitle%
    \end{minipage}%
}

%%% title page
% TODO: get rid of the navigation symbols on the title page.
%   actually, \frame[plain] *should* remove them...
\setbeamertemplate{title page}{
% upper right: FU logo
\vskip2pt\hfill\pgfuseimage{big-university-logo} \\
\vskip-20pt\hskip3pt
% title image of the presentation
\begin{minipage}{5.0cm}
\hspace{-.1cm}\inserttitlegraphic
\end{minipage}

% set the title and the author
\vskip-12pt
\parbox[top][1.35cm][c]{\textwidth}{\raggedleft{\color{text-grey}\inserttitle \\ \small \insertsubtitle}}
\vskip3pt
\parbox[top][1.35cm][c]{\textwidth}{\raggedleft{\small \insertauthor \\ \insertinstitute \\[3mm]
\scriptsize \insertdate}} }
%%% end title page

%%% colors
\usecolortheme{lily}
\setbeamercolor*{normal text}{fg=black,bg=white}
\setbeamercolor*{alerted text}{fg=usp-red}
\setbeamercolor*{example text}{fg=usp-orange}
\setbeamercolor*{structure}{fg=usp-blue}

\setbeamercolor*{block title}{fg=white,bg=black!50}
\setbeamercolor*{block title alerted}{fg=white,bg=black!50}
\setbeamercolor*{block title example}{fg=white,bg=black!50}

\setbeamercolor*{block body}{bg=black!10}
\setbeamercolor*{block body alerted}{bg=black!10}
\setbeamercolor*{block body example}{bg=black!10}

\setbeamercolor{bibliography entry author}{fg=usp-blue}
% TODO: this doesn't work at all:
\setbeamercolor{bibliography entry journal}{fg=text-grey}

\setbeamercolor{item}{fg=usp-blue}
\setbeamercolor{navigation symbols}{fg=text-grey,bg=bg-grey}
%%% end colors

%%% headline
\setbeamertemplate{headline}{
\vskip4pt\hfill\insertlogo\hspace{3.5mm} % logo on the right

\vskip6pt\color{usp-blue}\rule{\textwidth}{0.4pt} % horizontal line
}
%%% end headline

%%% footline
\newcommand{\footlinetext}{\insertshortinstitute, \insertshorttitle, \insertshortdate}
\setbeamertemplate{footline}{
\vskip5pt\color{usp-blue}\rule{\textwidth}{0.4pt}\\ % horizontal line
\vskip2pt
\makebox[123mm]{\hspace{7.5mm}
\color{usp-blue}\footlinetext
\hfill \raisebox{-1pt}{\usebeamertemplate***{navigation symbols}}
\hfill \insertframenumber}
\vskip4pt
}
%%% end footline

%%% settings for listings package
\lstset{extendedchars=true, showstringspaces=false, basicstyle=\footnotesize\sffamily, tabsize=2, breaklines=true, breakindent=10pt, frame=l, columns=fullflexible}
\lstset{language=Python} % this sets the syntax highlighting
\lstset{mathescape=true} % this switches on $...$ substitution in code
% enables UTF-8 in source code:
\lstset{literate={ä}{{\"a}}1 {ö}{{\"o}}1 {ü}{{\"u}}1 {Ä}{{\"A}}1 {Ö}{{\"O}}1 {Ü}{{\"U}}1 }
%%% end listings  % THIS is the line that includes the template!

\usepackage{arev,t1enc} % looks nicer than the standard sans-serif font
% if you experience problems, comment out the line above and change
% the documentclass option "9pt" to "10pt"


% ---------------------------------------------------------------------------- %
\DeclareMathOperator*{\argmin}{arg\,min}
\DeclareMathOperator*{\argmax}{arg\,max}
\DeclareMathOperator*{\erf}{erf}
% ---------------------------------------------------------------------------- %


% image to be shown on the title page (without file extension, should be pdf or png)

\titleimage{images/capa_pga}

\title[Smoothing Techniques] % (optional, use only with long paper titles)
{Técnicas de Suavização}

\subtitle
{aplicadas à caracterização de fontes sísmicas e \\ 
 à análise probabilística de ameaça sísmica (PSHA)}

\author[Pirchiner, Marlon] % (optional, use only with lots of authors)
{M.~Pirchiner }
% - Give the names in the same order as the appear in the paper.

\institute[EMAp-FGV / IAG-USP] % (optional, but mostly needed)
{EMAp-FGV / Centro de Sismologia - USP}
% - Keep it simple, no one is interested in your street address.

\date[EMAP2014] % (optional, should be abbreviation of conference name)
{julho de 2014}
%{Defesa de Mestrado, julho de 2014}
% - Either use conference name or its abbreviation.
% - Not really informative to the audience, more for people (including
%   yourself) who are reading the slides online

\subject{Seismology, Earthquake, Seismic Hazard, Smoothing}
% This is only inserted into the PDF information catalog. Can be left
% out.

% you can redefine the text shown in the footline. use a combination of
% \insertshortauthor, \insertshortinstitute, \insertshorttitle, \insertshortdate, ...
\renewcommand{\footlinetext}{\insertshortinstitute,\insertshorttitle,\insertshortdate}

% Delete this, if you do not want the table of contents to pop up at
% the beginning of each subsection:
\AtBeginSubsection{
  \begin{frame}<beamer>{Agenda}
    \tableofcontents[currentsection, currentsubsection]
  \end{frame}
  }


\begin{document}


\begin{frame}[plain]
  \titlepage
\end{frame}

\begin{frame}{Agenda}
  \tableofcontents
  % You might wish to add the option [pausesections]
\end{frame}


%=============================================================================
\section{Contexto}
%=============================================================================
%=============================================================================
\subsection{Tectônica e Tremores de Terra}
%=============================================================================

%-----------------------------------------------------------------------------
\begin{frame}{Distribuição mundial de epicentros}
\begin{figure}[H]
   \centering
   \includegraphics[height=0.95\textheight]{global_pde_mag_all}
   \caption[Mapa Mundial de Epicentros 1963-1998]
   		   {Mapa Mundial de Epicentros 1963-1998 - \citet{lowman_jr_1998}} 
   \label{f:global_epicenters}
\end{figure} 
\end{frame}


%-----------------------------------------------------------------------------
\begin{frame}{Teoria de Placas}
\begin{figure}[H]
   \centering
   \includegraphics[height=0.95\textheight]{litho_plates_overview}
   \caption[Cartografia das placas litosféricas]
   		   {Cartografia das placas litosféricas - \citet{usgs_plates_1996}} 
   \label{f:plates_overview}
\end{figure} 
\end{frame}



%=============================================================================
\subsection{Sismicidade da América do Sul}
%=============================================================================
%-----------------------------------------------------------------------------
\begin{frame}{Geologia e sismicidade da América do Sul}

\begin{columns}[T]
	\begin{column}[T]{0.5\textwidth}
		\begin{figure}[T]
		  \centering
		  \includegraphics[width=.85\textwidth]{lithology_sa} 
		  \caption{Mapa geológico (\url{http://onegeology.org}).}
		  \label{fig:sa_tec} 
		\end{figure}
	\end{column}
	\begin{column}[T]{0.49\textwidth}
		\begin{figure}[T]
		  \centering
		  \includegraphics[width=.85\textwidth]{seismicity_sa} 
		  \caption{Sismicidade, Catálogo ISC-GEM.}
		  \label{fig:sa_seis} 
		\end{figure}
	\end{column}
\end{columns}

\end{frame}



%=============================================================================
\subsection{Sismicidade do Brasil}
%=============================================================================

%-----------------------------------------------------------------------------
\begin{frame}{Geologia e sismicidade do Brasil}
\begin{figure}[H]
	\scriptsize
	\centering
	\begin{subfigure}[T]{0.48\textwidth}
	  \centering
	  \includegraphics[width=1.0\textwidth]{tectonico_brasil} 
	  \subcaption{Mapa Geológico do Brasil em escala 1:1.000.000.
	  \citet{bizzi_2003}}
	  \label{fig:br_tec} 
	\end{subfigure}
	\begin{subfigure}[T]{0.48\textwidth}
	  \centering
	  \includegraphics[width=1.0\textwidth]{seismicity_br} 
	  \subcaption{Sismicidade do Brasil. Catálogo \gls{bsb2013}.}
	  \label{fig:br_seis} 
	\end{subfigure}
	%\caption{Geologia e Sismicidade do Brasil}
	\label{fig:eq_record}
\end{figure}
\end{frame}



%=============================================================================
\section{Problema}
%=============================================================================
%=============================================================================
\subsection{Cálculo de ameaça sísmica}
%=============================================================================

%-----------------------------------------------------------------------------
\begin{frame}{Fluxo de trabalho}
\begin{figure}[H]
	\centering
	\begin{tabular}{l}
	\includegraphics[height=0.90\textheight]{classical_psha_workflow}
	\end{tabular}
	\caption{Fluxo de trabalho clássico para a \gls{psha} \citep{crowley_2013}.}
\label{fig:classical_psha}
\end{figure}
\end{frame}


%=============================================================================
\subsection{Método de Zoneamento (Cornell \& McGuire)}
%=============================================================================

\begin{frame}{Considerações sobre o método de zoneamento}

	\begin{itemize}
		\item em cada zona $i$, o processo de ocorrência de tremores seja 
		modelado como um processo de Poisson com taxa $\lambda_i$, assumindo que os tremores em diferentes zonas são
		independentes.
		\item Numa zona $i$, a magnitude dos tremores é modelada como uma \gls{va}
		com densidade $f_{M_i}(\cdot)$.
		\item A distância entre cada tremor da zona $i$ e o local $S$ é modelada como uma \gls{va}
		com densidade $f_{D_i}(\cdot)$.
		\item O modelo de predição do movimento do chão (\gls{gmpe}) é expresso pela regressão da medida de intensidade
		em magnitude, distância, condições geológicas do local $S$ e outros fatores.
	\end{itemize}

\end{frame}



%-----------------------------------------------------------------------------
\begin{frame}{Integral de Ameaça}
	\begin{block}{Probabilidade de excedência (método de zoneamento)}
		\begin{equation} 
		p_i= \displaystyle \int_{m_i =M_{\min}(i)}^{M_{\max}(i)}
		\int_{x_i =0}^{\infty}  P \left\{ \mbox{\small intensidade \normalsize }>I| M_i = m_i; D_i = x_i \right\}f_{M_i}(m_i) f_{D_i}( x_i )
		\mathrm{d}m_i \mathrm{d}x_i
		\end{equation}
	\end{block}
	onde $f_{M_i}$ é a distribuição de magnitudes na zona $i$, 
	 $f_{D_i}$ é a distribuição das distâncias entre o local de análise e a zona
	 $i$. $P \left\{ \mbox{\small intensidade \normalsize }>I| M_i = m_i; D_i = x_i
	\right\}$ é estimada pelos modelos de predição de movimento do chão (GMPEs).
\end{frame}


%-----------------------------------------------------------------------------
\begin{frame}{Equações de predição de movimento do chão}	
	\begin{block}{GMPE}
		\begin{equation} \label{pgamodel}
		\ln I = \overline{\ln I}(M, D, \theta) + \sigma(M, D, \theta) \varepsilon.
		\end{equation}
	\end{block}
	onde $\overline{\ln I}(M, D, \theta)$ (e respectivamente $\sigma(M, D,
	\theta)$) é a média condicional (e desvio padrão) de $\ln I$ para certa magnitude $M$, distância $D$
	e condições $\theta$, enquanto $\varepsilon$ é uma \gls{va} gaussiana padrão.
\end{frame}



%=============================================================================
\subsection{Ferramentas}
%=============================================================================
%-----------------------------------------------------------------------------
\begin{frame}{Openquake}
\begin{figure}[!h]
  \centering
  \includegraphics[height=.90\textheight]{oq_ecosystem} 
  \caption{Ecossistema de módulos, bibliotecas e utilitários do OpenQuake}
  \label{fig:oq} 
\end{figure}
\end{frame}


%-----------------------------------------------------------------------------
\begin{frame}{Openquake: principais componentes}
\begin{itemize}
  \item oq-engine: cálculo de ameaça e risco
  \item oq-hazardlib: biblioteca para o cálculo de ameaça sísmica
  \item oq-nrmllib: linguagem de marcação para ameaça e risco sísmico
  \item oq-risklib: biblioteca para o cálculo de risco sísmico
  \item \alert{HMTK}: kit de ferramentas para modelagem do cálculo de ameaça.
\end{itemize}
\end{frame}



%=============================================================================
\subsection{Resultados Anteriores}
%=============================================================================
%-----------------------------------------------------------------------------
\begin{frame}{GSHAP: Ameaça}
\begin{figure}[H]
  \centering
  \includegraphics[height=.95\textheight]{pga_gshap} 
  \caption{Resultado do GSHAP para o Brasil, PGA(10\%/50anos), em unidades de $g$}
  \label{fig:gshap} 
\end{figure}
\end{frame}


%-----------------------------------------------------------------------------
\begin{frame}{Dourado2014: Método de Zoneamento (Cornell \& McGuire)}
\begin{figure}[H]
  \centering
  \includegraphics[height=.93\textheight]{a_dourado} 
  \caption{Zoneamento sísmico e caracterização das zonas sísmicas por \citep{dourado_2014}.
  Os valores para a magnitude mínima foram de 3.0}
  \label{fig:a_dourado} 
\end{figure}
\end{frame}


%-----------------------------------------------------------------------------
\begin{frame}{Dourado2014: Ameaça}
\begin{figure}[H]
  \centering
  \includegraphics[height=.95\textheight]{pga_dourado_oq} 
  \caption{Mapa de ameaça sísmica, PGA(poe 0.1, 50y)[Dourado, 20014] OpenQuake-Engine }
  \label{fig:pga_dourado_oq} 
\end{figure}
\end{frame}



%=============================================================================
\section{Dados}
%=============================================================================
%=============================================================================
\subsection{Visão Geral}
%=============================================================================

%-----------------------------------------------------------------------------
\begin{frame}{Catálogos}
	\begin{columns}[T]
		\begin{column}[T]{.5\textwidth}
			\textbf{ISC-GEM}\\
			\small
			\begin{itemize}
				\item versão: 1.04
				\item licença: CC-BY-SA
				\item registros: 110.000 aprox.
				\item magnitudes: $\gls{sym:Mw}$
				\item característica: redeterminação de
			parâmetros
				\item fonte: \gls{iscgem}
			\end{itemize}
		\end{column}
		
		\begin{column}[T]{.5\textwidth}
			\textbf{BSB2013.08}\\
			\small
			\begin{itemize}
				\item versão: 2013.08
				\item licença: CC-BY
				\item registros: 900 aprox.
				\item magnitudes: maioria $mR$
				\item característica: determinação de parâmetros
				\item fonte: colaborativo (ObSis-UnB, Unesp, IPT, UFRN, ON, IAG-USP e outros).
			\end{itemize}
		\end{column}
	\end{columns}
\end{frame}


%-----------------------------------------------------------------------------
\begin{frame}{Distribuição de Profundidades}
\begin{figure}[H]
	\centering
	\begin{subfigure}[t]{0.45\textwidth}
		  	\centering
			\includegraphics[width=1.00\textwidth]{dep_sa_hist}
			\subcaption{Distribuição da profundidade dos tremores, \gls{iscgem}}
			\label{fig:sa_dep_hist}
	\end{subfigure}%
	\quad %~ %add desired spacing between images, e. g. ~, \quad, \qquad, \hfill etc.
	\begin{subfigure}[t]{0.4\textwidth}
		  	\centering
			\includegraphics[width=1.00\textwidth]{dep_br_hist}
			\subcaption{Distribuição da profundidade dos tremores, \gls{bsb2013}}
			\label{fig:br_dep_hist}
        \end{subfigure}%
%  \caption{Profundidades.}
  \label{fig:qc_histograms} 
\end{figure}
\end{frame}


%-----------------------------------------------------------------------------
\begin{frame}{Número de sismos por dias da semana}
\begin{figure}[H]
	\centering
	\begin{subfigure}[t]{0.48\textwidth}
		  	\centering
			\includegraphics[width=1.0\textwidth]{hmtk_sa3_weekday}
			\subcaption{Distribuição dos tremores nos dias da semana, \gls{iscgem}}
			\label{fig:sa_week_hist}
	\end{subfigure}%
	\quad %~ %add desired spacing between images, e. g. ~, \quad, \qquad, \hfill etc.
	\begin{subfigure}[t]{0.48\textwidth}
		  	\centering
			\includegraphics[width=1.0\textwidth]{hmtk_bsb2013_weekday}
			\subcaption{Distribuição dos tremores nos dias da semana, \gls{bsb2013}}
			\label{fig:br_week_hist}
    \end{subfigure}%
  %\caption{Checagem de qualidade.}
  \label{fig:qc_histograms} 
\end{figure}
\end{frame}



%-----------------------------------------------------------------------------
\begin{frame}{Distribuição da hora de ocorrência}
\begin{figure}[H]
	\centering
 	\begin{subfigure}[t]{0.45\textwidth}
		  	\centering
			\includegraphics[width=1.00\textwidth]{hmtk_sa3_hour}
			\subcaption{Distribuição do horário de ocorrência dos tremores, \gls{iscgem}}
			\label{fig:sa_hour_hist}
	\end{subfigure}%
	\quad %~ %add desired spacing between images, e. g. ~, \quad, \qquad, \hfill etc.
	\begin{subfigure}[t]{0.45\textwidth}
		  	\centering
			\includegraphics[width=1.00\textwidth]{hmtk_bsb2013_hour}
			\subcaption{Distribuição do horário de ocorrência dos tremores, \gls{bsb2013}}
			\label{fig:br_hour_hist}
    \end{subfigure}%
  %\caption{Checagem de qualidade.}
  \label{fig:qc_histograms} 
\end{figure}
\end{frame}




\begin{frame}{Distribuições de Frequência e Magnitude (MFD)}
	\begin{block}{Lei de Gutenberg-Richter}
		\begin{equation}
			\gls{eqn:gr_mfd}
			\label{eq:gr_mfd}
		\end{equation}
	\end{block}
	\small
	\glsdesc*{eqn:gr_mfd}.
\end{frame}


%-----------------------------------------------------------------------------
\begin{frame}{MFD: ISC-GEM e BSB2013.08}
\begin{figure}[H]
   \centering
   \includegraphics[height=0.90\textheight]{occurrence}
   \caption[Distribuição incremental e cumulativa de frequência e magnitude dos sismos presentes no catálogo ISC-GEM
   para a América do Sul unido com o \gls{bsb2013}]
   {Distribuição incremental e cumulativa de frequência e magnitude dos sismos presentes no catálogo ISC-GEM
   para a América do Sul unido com o \gls{bsb2013}} 
   \label{f:occurrence}
\end{figure} 
\end{frame}



%-----------------------------------------------------------------------------
\begin{frame}{Distribuição Incremental de Frequência e Magnitude}
Com as transformações $\alpha = 10^a$ e $\beta = b\ln{10}$, a mesma
distribuição pode ser escrita como: \\~\\

	\begin{block}{Gutenberg-Richter Incremental}
		\begin{equation}
			\begin{split}
				\gls{sym:N_m} &= 10^{\gls{sym:a} - \gls{sym:b}\gls{sym:m}} \\
							  &= \alpha e^{-\beta m}
			\end{split}
			\label{eq:gr_exp}
		\end{equation}
	\end{block}

\end{frame}


%-----------------------------------------------------------------------------
\begin{frame}{Distribuição Cumulativa de Frequência e Magnitude}
A distribuição \alert{cumulativa} pode ser obtida da seguinte maneira: 
	\begin{block}{Gutenberg-Richter Cumulativa}
		\begin{equation}
			\begin{split}
				N(m > m_{min}) &= \alpha \int\limits_{m_{min}}^{\infty}e^{-\beta m}\mathrm{d}m \\
							   &= \frac{\alpha}{\beta} e^{-\beta m} \\
							   &= \alpha_{cum} e^{-\beta m}.
			\end{split}
			\label{eq:gr_cum}
		\end{equation}
	\end{block}
	onde $\alpha_{cum} = \alpha / \beta $.
\end{frame}


%-----------------------------------------------------------------------------
\begin{frame}{MFD Duplamente truncada}
	\begin{block}{MFD Duplamente Truncada}
	\begin{equation}
			\gls*{sym:N_m} = \frac{e^{-\beta (m - m_{min})}}{1 -e^{-\beta (m_{max} - m_{min}) } } , m_{min} \leq m \leq m_{max}
		\label{eq:gr_dtr}
	\end{equation}
	\end{block}
	onde $m_{min}$ e $m_{max}$ são respectivamente as magnitudes mínima e máxima.

	\begin{itemize}
	  \item Uma das MFDs disponíveis e utilizadas no OpenQuake.
	\end{itemize}

\end{frame}


%-----------------------------------------------------------------------------
\begin{frame}{Ilustração das Distribuições de Frequência e Magnitude}
\begin{figure}[H]
   \centering
   \includegraphics[height=0.90\textheight]{mfd}
   \caption[Distribuições de frequência e magnitude]
   		   {Distribuições de frequência e magnitude} 
   \label{f:mfd}
\end{figure} 
\end{frame}



%-----------------------------------------------------------------------------
\begin{frame}{Taxa de sismicidade}
\begin{figure}[H]
	\centering
	\begin{subfigure}[t]{0.50\textwidth}
	  	\centering
		\includegraphics[width=1.0\textwidth]{hmtk_sa3_rate}
		\subcaption{Número de tremores registrados por ano, \gls*{iscgem}}
		\label{fig:sa_eq_record}
	\end{subfigure}%
	~ %~ %add desired spacing between images, e. g. ~, \quad, \qquad, \hfill etc.
	\begin{subfigure}[t]{0.50\textwidth}
	  	\centering
		\includegraphics[width=1.0\textwidth]{hmtk_bsb2013_rate}
		\subcaption{Número de tremores registrados por ano, \gls{bsb2013}}
		\label{fig:br_eq_record}
    \end{subfigure}%
	\caption{Número de tremores registrados por ano após 1900}
	\label{fig:eq_record}
\end{figure}
\end{frame}



%-----------------------------------------------------------------------------
\begin{frame}{Distribuição Temporal de Magnitudes}
\begin{figure}[H]
	\centering
	\includegraphics[height=0.41\textheight]{time_mag_count_sa}
	\caption{Catálogo \gls{iscgem} 1900-2012}
	\label{fig:tmf_sa}
\end{figure}
\begin{figure}[H]
	\centering
	\includegraphics[height=0.41\textheight]{time_mag_count_br}
	\caption{Catálogo \gls{bsb2013} 1900-2012}
	\label{fig:tmf_br}
\end{figure}%
\end{frame}





%=============================================================================
\subsection{Remoção de Agrupamentos}
%=============================================================================

%-----------------------------------------------------------------------------
\begin{frame}{Remoção de Agrupamentos: Métodos}
	A metodologia mais geral para identificação de agupamentos consiste em definir
	janelas no espaço e no tempo para cada magnitude e associar ao agrupamento
	sismos que ocorram dentro dessas janelas.
	\vspace{0.3cm}

	A primeira de formulação para o método de janelas foi a de 
	\begin{block}{\citet{gardner_1974}}
		\begin{equation}\begin{split} 
		\mbox{d(m)} = &10^{0.1238 m + 0.983}\\
		\mbox{t(m)} = & 
		\begin{cases} 10^{0.032 m + 2.7389} & \text{se $m \geq 6.5$} \\ 
		              10^{0.5409 m - 0.547} & \mbox{caso contrário}  \end{cases}\end{split}
		\end{equation}
	\end{block}

\end{frame}

%-----------------------------------------------------------------------------
\begin{frame}{Remoção de Agrupamentos: Métodos}
	Uma formulação alternativa foi apontada em \citep{marsan_david_2012} e proposta por
	\begin{block}{Gr\"unthal}
		\begin{equation}\begin{split} 
		\mbox{d(m)} = & e^{1.77 + \left( {0.037 + 1.02 m} \right)^2} \\ 
		   \mbox{t(m)} = & \begin{cases}   |e^{-3.95+ \left( {0.62 + 17.32 m}
		    \right)^2}|    & \text{se $m \geq 6.5$ } \\ 10^{2.8 + 0.024 m} & 
		    \text{caso contrário}  \end{cases}\end{split}
		\end{equation}
	\end{block}
	\vspace{0.3cm}
	E outra sugerida por 
	\begin{block}{\citet{uhrhammer_1986}}
		\begin{equation}
		\mbox{d(m)} = e^{-1.024 + 0.804 m} \quad \mbox{t(m)} = 
		    e^{-2.87 + 1.235 m}
		\end{equation}
	\end{block}
	\vspace{0.3cm}
	
	\begin{itemize}
		\item \citet{musson_1999}, propôs que as janelas de tempo
fossem janelas móveis em vez de fixas implementado sua proposta no programa AFTERAN.
	\end{itemize}
\end{frame}





%-----------------------------------------------------------------------------
\begin{frame}{Remoção de Agrupamentos: \gls{bsb2013}}
\begin{figure}[H]
	\centering
	\includegraphics[height=0.90\textheight]{decluster_br}
	\caption{Número cumulativo de tremores registrados por ano para o \gls{bsb2013}
	original e para diferentes métodos/janelas de remoção de agrupamentos.}
	\label{fig:br_eq_record}
\end{figure}
\end{frame}


%-----------------------------------------------------------------------------
\begin{frame}{Remoção de Agrupamentos: \gls{iscgem}}
\begin{figure}[H]
	\centering
	\includegraphics[height=0.90\textheight]{decluster_sa}
	\caption{Número cumulativo de tremores registrados por ano para o \gls{iscgem}
	original e para diferentes métodos/janelas de remoção de agrupamentos.}
	\label{fig:sa_eq_record}
	\end{figure}%
\end{frame}


%-----------------------------------------------------------------------------
\begin{frame}{Agrupamentos}
\begin{figure}[H]
	\centering
	\begin{subfigure}[t]{0.46\textwidth}
		  	\centering
			\includegraphics[width=1.00\textwidth]{hmtk_sa3_pp_decluster}
			\subcaption{Número de tremores registrados por ano, \gls{iscgem}}
			\label{fig:sa_decluster}
	\end{subfigure}%
	\quad %~ %add desired spacing between images, e. g. ~, \quad, \qquad, \hfill etc.
	\begin{subfigure}[t]{0.50\textwidth}
		  	\centering
			\includegraphics[width=1.00\textwidth]{hmtk_bsb2013_pp_decluster}
			\subcaption{Número de tremores registrados por ano, \gls{bsb2013}}
			\label{fig:br_decluster}
    \end{subfigure}%
	%\caption{Número de tremores registrados por ano após 1900}
	\label{fig:eq_decluster}
\end{figure}
\end{frame}



%======================================
\subsection{Magnitude de Completude}
%======================================

%-----------------------------------------------------------------------------
\begin{frame}{Método de \citet{stepp_1971}}

Sejam $\lambda_1, \lambda_2, \cdots, \lambda_n$ o número de sismos por unidade de tempo.
\vspace{0.3cm}
Assumindo que os tremores de certa faixa de magnitude têm distribuição de Poisson, a
	\begin{block}{da taxa de média de sismicidade $\lambda$ por unidade de tempo}
		\begin{equation}
			\ensuremath{
				\lambda =  \frac{1}{n}\sum_{i=1}^{n} \lambda_i.
			}
		\label{eq:mwaf}
		\end{equation}
	\end{block}
    Com variância $\sigma_{\lambda}^2 =  \lambda/n$, sendo $n$ o número de intervalos.
	\vspace{0.3cm}

	Com intervalos de 1 ano, $n$ intervalos é o período $T$ de observação e o 
	\begin{block}{desvio padrão}
		\begin{equation}
			\ensuremath{
				 \sigma_{\lambda} = \frac{\sqrt{\lambda}}{\sqrt{T}}.
			}
		\label{eq:mwaf}
		\end{equation}
	\end{block}

	\vspace{0.3cm}
	Com taxa de sismicidade estacionária, seu desvio padrão deve se comportar como $1/\sqrt{T}$ para 
tempos de observação $T$ crescentes.

\end{frame}




%-----------------------------------------------------------------------------
\begin{frame}{Método de Stepp: \gls{iscgem}}
\begin{figure}[H]
	\centering
	\includegraphics[height=0.90\textheight]{stepp_sa}
	\caption{Diagrama de Stepp para o \gls{iscgem} (\emph{declustered})}
	\label{fig:sa_stepp}
\end{figure}
\end{frame}


%-----------------------------------------------------------------------------
\begin{frame}{Método de Stepp: \gls{bsb2013}}
\begin{figure}[H]
	\centering
	\includegraphics[height=0.90\textheight]{stepp_br}
	\caption{Diagrama de Stepp para o \gls{bsb2013} (\emph{declustered})}
	\label{fig:br_stepp}
\end{figure}
\end{frame}



%-----------------------------------------------------------------------------
\begin{frame}[c]{Magnitudes de Completude}
	\begin{table}[h]
	  	\centering
	  	\footnotesize
		\begin{tabular}{l|*{11}{c}}
		$M_c$ & 3.0  & 3.5  & 4.0  & 4.5  & 5.0  & 5.5  & 6.0  & 6.5  & 7.0  & 7.5  & 8 \\  \hline
		Ano   & 1986 & 1986 & 1986 & 1960 & 1958 & 1958 & 1927 & 1898 & 1885 & 1885 & 1885   \\
		\end{tabular}
		\caption{Magnitude de completude, \gls{iscgem}}
		\label{tab:mc_sa}
	\end{table} \\
	\quad		\\
	\quad		\\
	\begin{table}[h]
	  	\centering
	  	\footnotesize
		\begin{tabular}{l|*{7}{c}}
		$M_c$ & 3.0  & 3.5  & 4.0  & 4.5  & 5.0  & 5.5  & 6.0  \\  \hline
		Ano   & 1980 & 1975 & 1975 & 1965 & 1965 & 1860 & 1860 \\
		\end{tabular}
		\caption{Magnitude de completude, \gls{bsb2013}}
		\label{tab:mc_br}
	\end{table}
\end{frame}



%=============================================================================
\section{Técnicas de Suavização}
%=============================================================================
%=============================================================================
\subsection{Teoria Geral}
%=============================================================================
%-----------------------------------------------------------------------------
\begin{frame}{Histograma}
	\begin{block}{Histograma 2D normalizado da taxa anual de sismicidade}
		\begin{equation}
		\scriptsize
			\ensuremath{
			\frac{\text{o número observado de sismos na célula}}
				 {\text{área/volume da célula} \times 
				  \text{número total de sismos observados}}
			/
			\text{tempo de observação em anos}
			}
		\label{eq:rate_count}
		\end{equation}
	\end{block}
\end{frame}


%-----------------------------------------------------------------------------
\begin{frame}{Suavização: modelo simplificado}
Para os $n$ pares (célula, taxa de sismicidade) $(x_1, R_1), (x_2, R_2), \cdots, (x_n, R_n)$
obtidos pela contagem anterior, considere um modelo para a taxa de sismicidade $R$ 
em uma determinada célula $x_i$ a partir dessa amostra dada pelo
	\begin{block}{Modelo simplificado}
		\begin{equation}
			\ensuremath{
				R(x_i) = \lambda(x_i) + \epsilon_i,\;\;\; i=1,\dots,n
			}
		\label{eq:rate_model}
		\end{equation}
	\end{block}
	onde os $\epsilon_i$ são \gls{va} não-correlacionadas que
	representam os erros tais que $E(\epsilon_i \arrowvert X = x_i) = 0$ 
	e a $Var(\epsilon_i \arrowvert X = x_i) = \sigma^2(x_i)$. 
	
	\begin{itemize}
	  \item $\lambda(x_i) = E(R_i \arrowvert X = x_i)$ é uma função de regressão.
	\end{itemize}
\end{frame}



%-----------------------------------------------------------------------------
\begin{frame}{Suavização: estimadores}
É possível definir um estimador ou suavizador linear $\hat{\lambda}$ para $\lambda$
como
	\begin{block}{Estimador para a taxa de sismicidade}
		\begin{equation}
			\ensuremath{
				\hat{\lambda}(x) = \sum_{i=1}^{n}w_i(x)\,R_i.
			}
		\label{eq:rate_estim}
		\end{equation}
	\end{block}
se para todo $x \in \mathbb{R}$ existe uma sequência de pesos $w_1(x),
w_2(x),\cdots,w_n(x)$ tais que $\sum_{i=1}^{n}w_i(x) = 1$.
\end{frame}



%-----------------------------------------------------------------------------
\begin{frame}{Funções de Núcleo}
Função de núcleo (\emph{Kernel}) é qualquer função par, contínua e limitada que satisfaz as seguintes
	\begin{block}{Propriedades}
		\centering
		(i) $\int \lvert K(\boldsymbol{r})\rvert \,\mathrm{d}\boldsymbol{r} < \infty $
		\;\;\;\;\;(ii) $\underset{ \lvert\boldsymbol{r} \rvert \to \infty }{\lim} 
						\lvert \boldsymbol{r} \, K(\boldsymbol{r})\rvert =0$ 
		\;\;\;\;\;(iii) $\int \! K(\boldsymbol{r})	\,\mathrm{d}\boldsymbol{r} = 1 $
	\end{block}
\end{frame}


%-----------------------------------------------------------------------------
\begin{frame}{Formas das Funções de Núcleo}
	Um \emph{kernel} pode assumir muitas formas funcionais, por exemplo 
	\begin{block}{Gaussiano}
		\begin{equation}
			\ensuremath{
				K_{gs}(\gls{sym:r}\arrowvert h) = \eta_1(h)
					e^{- \frac{\|\gls{sym:r}\|^2}
		 				 	  {2 h^2 }},
		 	}
		\label{eq:kernel_gs}
		\end{equation}
	\end{block}
ou
	\begin{block}{Lei de Potência}
		\begin{equation}
			\ensuremath{
				K_{pl}(\gls{sym:r}\arrowvert h) = 
					\frac{\eta_2(h)}
		 				 {\left( \|\gls{sym:r}\|^2 + h^2 \right)^{\frac{3}{2}} },
		 	}
		\label{eq:kernel_pl}
		\end{equation}
	\end{block}
	onde $h$ é conhecida como \alert{largura de banda} ou
	\emph{bandwidth} e $\eta_1(h)$ e $\eta_2(h)$ são fatores de normalização para
	que as integrais sejam igual à unidade.
\end{frame}


%-----------------------------------------------------------------------------
\begin{frame}{Estimador de Nadaraya-Watson}
	Sejam $h \in \mathbb{R}, h > 0$ e $K$ uma função de núcleo.
	O estimador de Nadaraya-Watson (\citet{nadaraya_1965}) é definido pelos seguintes
	\begin{block}{Pesos de Nadaraya-Watson}
		\begin{equation}
			\ensuremath{
				w_i(x) = \frac{ K\left( \frac{x - x_i}{h} \right)}
							  {\sum_{j=1}^{n} K\left( \frac{x - x_j}{h} \right) },
			}
		\label{eq:rate_wi}
		\end{equation}
	\end{block}
\end{frame}


%=============================================================================
\subsection{\citet{frankel_1995}}
%=============================================================================

%-----------------------------------------------------------------------------
\begin{frame}{Frankel1995: Fundamento}
Consiste em aplicar o
\begin{block}{Suavizador gaussiano de Nadaraya-Watson}
\begin{equation}
	\ensuremath{
		\tilde{n}_j = \frac{ \sum_{i} n_i \,e^{ - \left(\frac{\gls{sym:dij}}{\gls{sym:dF}}\right)^2}}
						   { \sum_{i}     e^{ - \left(\frac{\gls{sym:dij}}{\gls{sym:dF}}\right)^2}},
	}
	\label{eq:ni}
\end{equation}
\end{block}
onde 
\begin{itemize}
	\item $d_F$ é a largura de banda \alert{fixa}, nomeada distância de
	correlação
	\item $\tilde{n}_j$ é a taxa cumulativa de sismicidade (número de sismos com magnitude
	$m$ maior que a mínima magnitude \gls{sym:Md} do catálogo) suavizada na célula $j$
	\item $n_i$ é o número de sismos em cada outra célula $i$ e
	\item \gls{sym:dij} é \glsdesc{sym:dij}.
\end{itemize}
\end{frame}


%-----------------------------------------------------------------------------
\begin{frame}{Frankel1995: Taxa de Sismicidade (\gls{bsb2013})}
\begin{figure}[H]
  \centering
  \includegraphics[height=.95\textheight]{a_frankel_br} 
  \caption{Mapa do valor-a, usando o catálogo \gls{bsb2013} calculado pelo método de Frankel, 1995 }
  \label{fig:a_fran_br} 
\end{figure}
\end{frame}



%-----------------------------------------------------------------------------
\begin{frame}{Frankel1995: Taxa de Sismicidade (\gls{iscgem})}
\begin{figure}[H]
  \centering
  \includegraphics[height=.95\textheight]{a_frankel_sa} 
  \caption{Mapa do valor-a, catálogo \gls{iscgem} [Frankel, 1995] }
  \label{fig:a_fran_sa} 
\end{figure}
\end{frame}


%-----------------------------------------------------------------------------
\begin{frame}{Frankel1995: Ameaça}
\begin{figure}[H]
  \centering
  \includegraphics[height=.95\textheight]{pga_frankel} 
  \caption{Mapa de ameaça sísmica, PGA (poe 10\%, 50y) [Frankel, 1995] }
  \label{fig:pga_fran} 
\end{figure}
\end{frame}





%=============================================================================
\subsection{\citet{woo_1996}}
%=============================================================================


%-----------------------------------------------------------------------------
\begin{frame}{Woo1996: Fundamento}
	\begin{block}{Taxa de Sismicidade}
		\begin{equation}
			\ensuremath{
				\gls{sym:Rrm} = \sum_{i=1}^{N} \frac{ K(\gls{sym:r} - \gls{sym:ri}, m)}
													{T({\gls{sym:ri})}},
			}
			\label{eq:Rrm}
		\end{equation}
	\end{block}
	onde $N$ é o número de tremores $i$ no catálogo 
	e $T(\gls{sym:ri})$ é o período em que todo sismo de magnitude acima de $m$ é completamente observado 
	em \gls{sym:ri}
\end{frame}


%-----------------------------------------------------------------------------
\begin{frame}{Woo1996: Funções de Núcleo}
	\begin{block}{\citet{kagan_knopoff_1980} - infinito}
		\begin{equation}
			\ensuremath{
				K_{KJ}(\gls{sym:r}, m \arrowvert \gls{sym:aW}) =  \frac{  \gls{sym:aW}  -1}{\pi\gls{sym:hm}^2}
									\left( 1 + \frac{\gls{sym:r}^2}{\gls{sym:hm}^2} \right)^{-\gls{sym:aW}},
			}
			\label{eq:k_kj}
		\end{equation}
	\end{block}
	onde \gls{sym:aW} é \glsdesc{sym:aW}.

	\begin{block}{\citet{verejones_1992} - finito}
		\begin{equation}
			\ensuremath{
				K_{VJ}(\gls{sym:r}, m \arrowvert \gls{sym:DW}) = 
				\begin{cases}
					\frac{\gls{sym:DW}}{2\pi\,\gls{sym:hm}} 
					\left( \frac{\gls{sym:hm}}{\gls{sym:r}} \right)^{2 - \gls{sym:DW}} 
					  & \gls{sym:r} \leq \gls{sym:hm} \\
					0 & \gls{sym:r} > \gls{sym:hm}
				\end{cases},
			}
			\label{eq:k_vj}
		\end{equation}
	\end{block}
	onde \gls{sym:DW} é \glsdesc{sym:DW}.
\end{frame}



%-----------------------------------------------------------------------------
\begin{frame}{Woo1996: Largura de banda}
A largura de banda no método de Woo, \alert{depende da magnitude}.
\begin{block}{Largura de banda dependente da magnitude}
	\begin{equation}
		\ensuremath{
			h(m\arrowvert \gls{sym:a0}, \gls{sym:a1}) = \gls{sym:a0}e^{\gls{sym:a1}m},
		}
		\label{eq:hm}
	\end{equation}
\end{block}
$a_0$ e $a_1$ são determinados pela regressão entre a 
distância média $h$ de cada tremor ao vizinho mais próximo em cada faixa de magnitude $m \pm \mathrm{d}m$
\end{frame}


%-----------------------------------------------------------------------------
\begin{frame}{Woo1996: Ajuste da largura de banda}
\begin{figure}[H]
  \centering
  \includegraphics[width=.95\textwidth]{woo_bandwidth} 
  \caption{Ajuste da largura de banda para o método de Woo1996}
  \label{fig:woo_b} 
\end{figure}
\end{frame}


%-----------------------------------------------------------------------------
\begin{frame}{Woo1996: Taxa de Sismicidade (\gls{bsb2013})}

\begin{figure}[H]
  \centering
  \includegraphics[height=.95\textheight]{a_woo} 
  \caption{Mapa do valor-a, usando o catálogo \gls{bsb2013} calculado pelo método de Woo, 1996 }
  \label{fig:a_woo} 
\end{figure}

\end{frame}



%-----------------------------------------------------------------------------
\begin{frame}{Woo1996: Ameaça (MFD discreta e incremental)}
\begin{figure}[H]
  \centering
  \includegraphics[height=.95\textheight]{pga_woo_inc} 
  \caption{Mapa de ameaça sísmica, PGA (poe 10\%, 50y).}
  \label{fig:pga_woo_inc} 
\end{figure}
\end{frame}



%-----------------------------------------------------------------------------
\begin{frame}{Woo1996: Ameaça (MFD truncada)}
\begin{figure}[H]
	\centering
	\includegraphics[height=0.95\textheight]{pga_woo_cum} 
	\caption{Mapa de ameaça sísmica, PGA (poe 10\%, 50y).}
	\label{fig:pga_woo_cum} 
\end{figure}
\end{frame}



%-----------------------------------------------------------------------------
\begin{frame}{Woo1996: Ameaça (diferença)}
\begin{figure}[H]
	\centering
		\includegraphics[height=0.95\textheight]{pga_woo_dif} 
		\caption{Mapa diferencial de ameaça, PGA (poe 10\%, 50y)
		   entre os mapas das figuras 
		   \ref{fig:pga_woo_inc} e \ref{fig:pga_woo_cum}.}
		\label{fig:pga_woo_dif} 
\end{figure}
\end{frame}



%=============================================================================
\subsection{\citet{helmstetter_2012}}
%=============================================================================

%-----------------------------------------------------------------------------
\begin{frame}{Helmstetter2012: Fundamentos}
	\begin{block}{Taxa de sismicidade}
		\begin{equation}
			\ensuremath{\gls{sym:R} = \sum_{i=1}^{N}{ \frac{1}{h_i\,{d_i}^2} \gls{sym:Kt}\gls{sym:Kr} }},
			\label{eq:helms01}
		\end{equation}
	\end{block}
	onde \gls{sym:R} é \glsdesc{sym:R}, 
		  $K_t$ é a \glsdesc{sym:Kt}, 
		  $K_r$ é a \glsdesc{sym:Kr}.
\end{frame}


%-----------------------------------------------------------------------------
\begin{frame}{Helmstetter2012: Fundamentos}
Restringindo os tempos do catálogo aos $t_i
< t$ e ponderando segundo um peso $w$:
	\begin{block}{Modelo para o \emph{forecast} da taxa de sismicidade}
		\begin{equation}
		\ensuremath{\gls{sym:R} = \gls{sym:Rmin} + \sum_{t_i < t}{ 
			\frac{2\,w(\boldsymbol{r}_i,t_i)}{h_i\,{d_i}^2}
					\gls{sym:Kt}\gls{sym:Kr} }},
			\label{eq:helms02}
		\end{equation}
	\end{block}
	onde onde \gls{sym:Rmin} é a \glsdesc{sym:Rmin} e os

	\begin{block}{Pesos}
		\begin{equation}
			\ensuremath{ w(\boldsymbol{r},t) = 10^{ \gls{sym:b}(\boldsymbol{r},t) \left[ \gls{sym:Mc_rt} - \gls{sym:Md}
			\right] } },
			\label{eq:helms_wi}
		\end{equation}
	\end{block}

	onde \gls{sym:wi} é o \glsdesc{sym:wi} na localização $\boldsymbol{r}$ e no instante $t$, 
		  \gls{sym:b}$(\boldsymbol{r},t)$ é o \glsdesc{sym:b}, 
		  \gls{sym:Mc_rt} é a \glsdesc{sym:Mc_rt}, 
		  \gls{sym:Md} é a \glsdesc{sym:Md}.

\end{frame}


%-----------------------------------------------------------------------------
\begin{frame}{Helmstetter2012: Método KNN-acoplado}
	A determinação das larguras de banda \alert{localmente adaptáveis} segundo a densidade
	espacial e temporal para cada um dos eventos é fruto de uma adaptação da 
	otimização proposta por \citet{choi_1999}:
	\begin{block}{Método acoplado dos vizinhos mais próximos}
		\begin{equation}
			\ensuremath{
		%		h_i, d_i = \underset{d_i \ge \gls{sym:dk}, h_i \ge \gls{sym:hk}}{\argmin} 
				h_i, d_i = \argmin_{\substack{h_i \ge \gls{sym:hk} \\
								              d_i \ge \gls{sym:dk}}
						           } 
				\left[ s \left(h_i,d_i 
					 		  \arrowvert
							  \gls{sym:k_cnn},\gls{sym:a_cnn}
					     \right) 
					   := h_i + \gls{sym:a_cnn}d_i 
			    \right],
			}
			\label{eq:helms_cnn}
		\end{equation}
	\end{block}
	onde \gls{sym:k_cnn} é o \glsdesc{sym:k_cnn},
		 \gls{sym:a_cnn} é o \glsdesc{sym:a_cnn},
		 \gls{sym:dk} é o \glsdesc{sym:dk} e 
		 \gls{sym:hk} é o \glsdesc{sym:hk}.
\end{frame}


%-----------------------------------------------------------------------------
\begin{frame}{Helmstetter2012: Larguras de banda}
\begin{figure}[H]
  \centering
  \includegraphics[height=.90\textheight]{helmstetter_hidi} 
  \caption{Exemplo da largura de banda para um determinado evento para o método de Helmstetter, com $k_{cnn} = 5$ e
  $a_{cnn} = 100$}
  \label{fig:h_hidi} 
\end{figure}
\end{frame}


%-----------------------------------------------------------------------------
\begin{frame}{Helmstetter2012: Taxa de sismicidade estacionária}
	É calculada a partir da mediana da taxa modelada para cada célula.
	\begin{block}{Taxa de sismicidade estacionária}
		\begin{equation}
			\ensuremath{
				\bar{R}(\boldsymbol{r}_0) = \text{Mediana}\left[R(\boldsymbol{r}_0, t)\right].
			}
			\label{eq:helms_mediana}
		\end{equation}
	\end{block}
	\begin{itemize}
		\item dispensa a aplicação do método de remoção de agrupamentos.
	\end{itemize}
\end{frame}


%-----------------------------------------------------------------------------
\begin{frame}{Helmstetter2012: Taxa estacionária de sismicidade}
\begin{figure}[H]
  \centering
  \includegraphics[width=.90\textwidth]{helmstetter_stationary_a} 
  \caption{Taxa de sismicidade estacionaria calculada a partir da mediana da
  taxa de sismicidade modelada pelo método de Helmstetter2012 para uma determinada célula $r_0$}
  \label{fig:h_stationary} 
\end{figure}
\end{frame}


%-----------------------------------------------------------------------------
\begin{frame}{Helmstetter2012: Parâmetros}
	Os parâmetros que definem cada modelo são
	\begin{itemize}
		\item a taxa mínima de sismicidade \gls{sym:Rmin},
		\item \gls{sym:k_cnn} e 
		\item \gls{sym:a_cnn}
	\end{itemize}
	
	\vspace{0.8cm}
	
	E são escolhidos de forma a maximizar a capacidade 'preditiva' do modelo,
	que pode ser feita separando o catálogo em duas partes:
	\begin{itemize}
		\item uma é usada para definir o modelo
		\item outra é separada para avaliação do resultado.
	\end{itemize}
\end{frame}


%-----------------------------------------------------------------------------
\begin{frame}{Helmstetter2012: Catálogos de treinamento e de teste}
\begin{figure}[H]
  \centering
  \includegraphics[height=.90\textheight]{helmstetter_catalogues} 
  \caption{Catálogos de aprendizado e de teste para o método de \citet{helmstetter_2012}}
  \label{fig:h_catalogue} 
\end{figure}
\end{frame}


%-----------------------------------------------------------------------------
\begin{frame}{Helmstetter2012: Verossimilhança}
	A probabilidade de se observar exatamente $n$ eventos em um processo de Poisson
	com taxa $N_p$ é
		\begin{equation}
			\ensuremath{
				\gls{sym:pNn} = \frac{{N_p}^n e^{-N_p}}{n!}.
			}
			\label{eq:loglik}
		\end{equation}

	\vspace{0.5cm}
	E o logarítmo, a ser maximizado, da verossimilhança entre 
	o que o modelo predisse e o que foi observado 
	pode ser escrito como
	\begin{block}{Log-Verossimilhança}
		\begin{equation}
			\ensuremath{
				\gls{sym:L} = \sum_{i_x=1}^{N_x}\sum_{i_y=1}^{N_y}\log p\left[  \gls{sym:Np}, \gls{sym:nxy}  \right]
			}
			\label{eq:loglik}
		\end{equation}
	\end{block}
	onde \gls{sym:Np} é \glsdesc{sym:Np},
		\gls{sym:nxy} é \glsdesc{sym:nxy}.
\end{frame}


%-----------------------------------------------------------------------------
\begin{frame}{Helmstetter2012: Distribuição uniforme}
	No caso de uma distribuição de Poisson espacialmente uniforme com taxa $N_u$, o
	\begin{block}{Log-Verossimilhança (uniforme)}
		\begin{equation}
			\ensuremath{
				\gls{sym:Lu} = -N_t + 
				\sum_{i_x = 1}^{N_x}\sum_{i_y=1}^{N_y}
				\gls{sym:nxy}\log N_u - \log \left[ \gls{sym:nxy}! \right]
			}
			\label{eq:lu}
		\end{equation}
	\end{block}
	onde $N_u = N_t/N_c$ com $N_t$ o número de sismos no catálogo-alvo e $N_c$ o número de células.
\end{frame}

%-----------------------------------------------------------------------------
\begin{frame}{Helmstetter2012: Ganho de probabilidade}
	\citet{kagan_knopoff_1977} definiram o 
	\begin{block}{Ganho de probabilidade}
		\begin{equation}
			\ensuremath{
				\gls{sym:G} = e^{ \frac{\gls{sym:L} - \gls{sym:Lu}}{\gls{sym:Nt}}   }
			}
			\label{eq:gain}
		\end{equation}
	\end{block}
	por tremor predito por um determinado modelo sobre uma distribuição espacialmente uniforme
	da taxa de sismicidade.
\end{frame}

%-----------------------------------------------------------------------------
\begin{frame}{Helmstetter2012: Ganho de modelos}
	Se dois modelos são preparados para um mesmo catálogo-alvo com $N_t$ eventos,
	o ganho pode ser simplificada para
	\begin{block}{Ganho}
		\begin{equation}
			\ensuremath{
			\begin{align}
				G & = e^{\sum_{i = 1}^{N_t}
							\frac{\log \left[  \gls{sym:Npi} / \gls{sym:Nu}  \right]}
								 {\gls{sym:Nt}}
					  } \\
				  & = {\langle  \gls{sym:Npi} / \gls{sym:Nu}  \rangle}_{geom}
			\end{align}}
			\label{eq:G}
		\end{equation}	
	\end{block}
	
	Uma vez que 
	\begin{equation}
		\ensuremath{
		\begin{align}
			L - L_u & = \sum_{i_x = 1}^{N_x}\sum_{i_y=1}^{N_y}
					  \gls{sym:nxy}\log \left[ \frac{\gls{sym:Np}}{\gls{sym:Nu}} \right] \\
					& = \sum_{i = 1}^{N_t}\log \left[\frac{\gls{sym:Npi}}{\gls{sym:Nu}} \right]
		\end{align}}
		\label{eq:llu}
	\end{equation}
\end{frame}


%-----------------------------------------------------------------------------
\begin{frame}{Helmstetter2012: Parâmetros otimizados}
\begin{table}[H]
	\centering
	\begin{tabular}{c|c}
		Parâmetro & Valor \\ \hline
		$R_{min}$ & $0.1\times10^{-13}$ \\
		$a_{cnn}$ & 325 \\
		$k_{cnn}$ & 1 \\ \hline
		Ganho	  & 2.43
	\end{tabular}
	\caption{Parâmetros otimizados e Ganho para o método de Helmstetter a partir do catálogo \gls{bsb2013}}
	\label{tab:hemlstetter}
\end{table}
\end{frame}


%-----------------------------------------------------------------------------
\begin{frame}{Helmstetter2012: Taxa de sismicidade}
\begin{figure}[H]
  \centering
  \includegraphics[height=.95\textheight]{a_helmstetter} 
  \caption{Mapa do valor-a, usando o catálogo \gls{bsb2013} calculado pelo método de Helmstetter, 2012 }
  \label{fig:helm_r} 
\end{figure}
\end{frame}


%-----------------------------------------------------------------------------
\begin{frame}{Helmstetter2012: Ameaça}
\begin{figure}[H]
  \centering
  \includegraphics[height=.93\textheight]{pga_helmstetter} 
  \caption{Mapa de ameaça sísmica, PGA (poe 10\%, 50y), 
  		   calculado com o OpenQuake a partir das fontes sísmicas
  		   determinas pelo método de Helmstetter,2012 }
  \label{fig:helm_h} 
\end{figure}
\end{frame}




%=============================================================================
\section{Discussão}
%=============================================================================
%=============================================================================
\subsection{Resultados consolidados}
%=============================================================================

%-----------------------------------------------------------------------------
\begin{frame}{Painel de Resultados}

\begin{columns}
	\begin{column}[T]{.24\textwidth}
		\includegraphics[width=1\textwidth]{pga_gshap} \\
		\includegraphics[width=1\textwidth]{pga_dourado_oq}
	\end{column}
	
	\begin{column}[T]{.24\textwidth}
		\includegraphics[width=1\textwidth]{a_frankel_br} \\
		\includegraphics[width=1\textwidth]{pga_frankel}
	\end{column}
	
	\begin{column}[T]{.24\textwidth}
		\includegraphics[width=1\textwidth]{a_woo} \\
		\includegraphics[width=1\textwidth]{pga_woo_inc}
	\end{column}
	
	\begin{column}[T]{.24\textwidth}
		\includegraphics[width=1\textwidth]{a_helmstetter} \\
		\includegraphics[width=1\textwidth]{pga_helmstetter}
	\end{column}
	
\end{columns}
\end{frame}


\begin{frame}{Comentários}
\begin{itemize}
	\item Foi possivel aplicar as técnicas de suavização para gerar uma grade regular de fontes sísmicas 
	pontuais, e então calcular a ameaça sísmica obtendo valores razoáveis, 
	sem com isso definir zonas sísmicas, é o que se conhece também como métodos de \emph{zoneless}.

	\item As diferenças nas formas de escolher a largura de banda das funções de núcleo de cada método,
	podem talvez render ao método de Hemlstetter alguma vantagem, por ser localmente adaptável,
	quando define claramente feições no nordeste, mas outros métodos, com escolhas mais rígidas,
	também o fizeram, com bons resultados em outras regiões.
	
	\item O ajuste do método de Woo para o Brasil forneceu enormes larguras de banda, com mais de 3000km para
	as maiores magnitudes (são poucas e o método se baseia em vizinhos mais próximos), isso dilui
	a influência dos grandes sismos nas taxas de sismicidade.
	
	\item No caso do Brasil, essas primeiras estimativas sugerem que estudos de maior detalhe 
	sejam feitos nas regiões de maior destaque.
\end{itemize}
\end{frame}



%=============================================================================
\subsection{Considerações}
%=============================================================================

%-----------------------------------------------------------------------------
\begin{frame}{Considerações}
	\begin{itemize}
		\item mesmo com implementações e fundamentos relativamente distintos, os métodos, no geral, conseguiram apresentar
	uma boa estimativa das características da sismicidade a partir da sismicidade histórica catalogada.
		\item uns distribuiram mais a taxa de sismicidade e a ameaça, uns destacando certas feições, outros outras.
		\item a caracterização da MFD, o uso da sua forma discreta ou truncada, afeta o valor calculado para a
	ameaça e merece atenção.
		\item nenhum modelo parece ser o mais correto, todos mereçam devida consideração.
		\item não é possível negligenciar a ocorrência de sismos profundos da zona de subducção
	para a ameaça sísmica do extremo oeste do país.
		\item o openquake cumpriu seu papel como calculador de ameaças.
		\item os resultados do openquake foram compatíveis com o
	modelo global, mesmo os valores calculados com o mesmo modelo de fontes sísmicas.
		\item o HMTK se mostrou uma ferramenta essencial para a modelagem da ameaça e muitas funcionalidades permitem
		facilmente a implementação dos principais fluxos de trabalho.
	\end{itemize}
\end{frame}


%=============================================================================
\subsection{Desenvolvimentos futuros}
%=============================================================================
%-----------------------------------------------------------------------------
\begin{frame}{Desenvolvimentos futuros}
\begin{itemize}
	\item determinar a variação temporal e espacial da magnitude de completude. \citet{vorobieva_2013} aponta caminhos muito
	interessantes.
	\item com magnitudes de momento sísmico é possível avaliar modelos que levem em consideração a distribuição
	de tensões crustais e mesmo a suavização da distribuição espacial de esforços crustais.
	\item avançar na direção de critérios de seleção das relações de atenuação, outro braço do modelo de ameaças.
	\item estudos de desagregação \citep{pagani_2007} também precisarão ser feitos no futuro.
	\item avaliação das incertezas
	\item elaboração de um cenário de ameaça, detalhando a árvore lógica para as fontes sísmicas e movimento do chão. 
	\item modelos des suavização estudados funcionam bem para espressar a taxa estacionária de sismicidade, mas
	talvez seja possível estudar modelos onde a taxa de sismicidade varia com o tempo como os modelos de sequência epidêmica
	de pós-abalos (ETAS), que podem dar melhor resposta com pouco volume de dados.
	\end{itemize}
\end{frame}


\section*{Sumário}

\begin{frame}{Sumário}
	\begin{itemize}
	  \item Cálculo de ameaça sísmica e o papel fundamental da caracterização de fontes.
	  \item Caracterização da distribuição de frequencia e magnitude e taxa de sismicidade.
	  \item Metodologia clássica de zoneamento (\citet{cornell_1968, mcguire_1976})
	  \item Técnicas de suavização para cálculo da taxa de sismicidade.
	  \item Pré-processamento (controle de qualidade, agrupamentos e completude)
	  \item Cálculo de ameaça (OpenQuake)
	\end{itemize}
\end{frame}


% All of the following is optional and typically not needed. 
\appendix
\section<presentation>*{\appendixname}
\subsection<presentation>*{Referências}



\begin{frame}[allowframebreaks]{Referências}
	\scriptsize
	% ---------------------------------------------------------------------------- %
	% Bibliografia
	\backmatter \singlespacing   				% espaçamento simples
	%\bibliographystyle{plain} 	% citação bibliográfica textual
	\bibliographystyle{styles/plainnat-ime} 	% citação bibliográfica textual
	\bibliography{bib/bibliografia}  			% associado ao arquivo: 'bibliografia.bib'

\end{frame}


\end{document}

